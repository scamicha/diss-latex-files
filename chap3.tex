\begin{spacing}{1.0}
\chapter[Comparative Study of Conditions]{Comparative Study of Physical and Numerical Conditions}\label{chap:compare}
\end{spacing}
\thispagestyle{plain}
\begin{spacing}{0.9}
\begin{flushright}
{\it \footnotesize \ldots the steady motion was unstable for large disturbances\\ long before the critical velocity was reached,\\ a fact which agreed with the full-blown manner\\ in which the eddies appeared\\}
 {\small -- O. Reynolds, 1883}
\end{flushright}
\end{spacing}
To obtain a better understanding of the effects of various physical and numerical parameters in the standard version of the IUHG code described in \S \ref{sec:standard}, I performed a series of simulations with varied physical and numerical specifications. In this chapter, I describe the outcome of altering the following: initial surface density profile, azimuthal resolution, adiabatic index in the equation of state, and the amplitude of the initial random perturbation. I investigate the effect of varying each parameter on GI unstable disks in general, as well as examining some effects specific to each of the parameters. 

For the initial surface density profile study, I looked at three different disks with initial surface density profiles $\Sigma \sim \varpi^{-q}$ for $q = 0.5$, 1, and 1.5, each of which I started from an equilibrium disk and evolved for several thousand years. The impetus behind this study is to understand how varying the initial surface profile affects several aspects of disk evolution in an unstable disk. I analyzed whether this changed the timing or onset of each of the phases in the prototypical evolution of a unstable disk. I also studied the effect on some of the primary characteristics of the GIs, that is, the dominant unstable mode, the torques generated by the GIs, and the resulting mass transport. Finally, I looked at the surface density profile after a few thousand years of evolution in the asymptotic phase to gain insight into how GIs reorder disk structure. 

An important question when interpreting results from any fixed grid simulation is whether the grid resolution is influencing the phenomenon being studied. If so, numerical artifacts may either be masking the phenomenon or masquerading as a physical effect. To determine the resolution needed for the numerical convergence in the simulations, I interpolated one of the initial surface density profile comparison disks to several different azimuthal resolutions near the beginning of the asymptotic phase of the disk evolution. I tested the convergence of several physical phenomena including GI amplitude, the strength of torques generated by GIs, and the dominant unstable mode.

To study the effects of altering the polytropic index in the polytropic equation of state, I performed simulations with $\gamma = 7/5$ and additional simulations with $\gamma = 5/3$, and compared them to a disk from the initial surface density profile comparison. I analyzed these disks to see if the different equation of state would alter the strength of the GIs. I performed simulations with both $\gamma = 5/3$ and $\gamma = 7/5$ to characterize the fragmentation limit for each of the adiabatic indices. The fragmentation limit is the volumetric cooling time at which a disk will fragment, $t_{cool \; crit}$. Each of these additional simulations were started from an equilibrium disk model used in the initial surface density profile comparison. 

Each of the initial equilibrium disks is typically seeded with a random perturbation to allow the GIs to grow from noise. In most simulations, the amplitude of the perturbation used is rather small; however, there are astrophysical situations where an external perturbation could be more substantial. For example, the infall rate from the envelope surrounding a protoplanetary disk could fluctuate in both time and space. To study how a stronger initial perturbation can effect disk evolution, I performed simulations with an initial random perturbation using three different amplitudes.

Understanding the importance of each of these effects allowed me to make a more informed decision on the types of disks to simulate in subsequent studies where I examined more complicated physical effects. This chapter begins with a description of the various analysis techniques used in studying and comparing the disks. These techniques include computing the amplitude of the global Fourier amplitude $A_m$ for different numbers of spiral arms, determining the periodicity of discrete spiral modes, and calculating the gravitational torque due to the spiral structure generated by the GIs. The description of the methods is followed by discussion of some relevant background information on current debates as to the nature and importance of GIs and their effects.

\section{Analysis Procedures}\label{sec:IC:anal}

In this section I describe the tools used to compare and contrast various disk simulations. There are several trivially simple analyses performed, such as computation of the surface density profile, Toomre $Q$, midplane and meridional density maps, etc. which I will not elucidate here. In addition to these straightforward tasks, I used several more complex techniques to gain insight into the disk evolution, such as measuring the amplitude of the global Fourier components of the GIs, calculating the periodicity of the Fourier modes based on Fourier phase information, and computing global gravitational torques due to spiral structures induced by GIs.
 
\subsection{Global Fourier Components}\label{sec:IC:am}

In order to quantify the global amplitude of non-axisymmetric structure, I use the quantity $A_m$ defined by \citet{imamura2000},
\begin{equation}
A_m = \frac{\left (a_m^2 + b_m^2\right )^{1/2}}{\pi \int \rho_0 \varpi\, \dif \varpi\, \dif z}.
\label{eq:IC:am}
\end{equation}
Here,
\begin{subequations}\label{eq:IC:fourier}
\begin{align}
a_m&=\int_{V} \rho \cos (m\phi) \varpi\, \dif \varpi \,\dif z\, \dif \phi \label{eq:IC:cos}\\
\intertext{and}
b_m&=\int_{V} \rho \sin (m\phi) \varpi \,\dif \varpi\, \dif z \,\dif \phi, \label{eq:IC:sin}
\end{align}
\end{subequations} 
are the Fourier components and $\rho_0$ is the axisymmetric component of the density, and is defined as $\frac{1}{2\pi} \int_{-\pi}^{\pi} \rho \dif \phi$. The volume integrals are over the entire simulation volume. This analysis, by definition, neglects the phase information in the density variations. Values of $A_m$ are computed up to $m = l_{max}/2$, where $l_{max}$ is the azimuthal resolution. Higher $m$-values violate the Nyquist-Shannon theorem \citep{shannon1984}. Due to the fact that the simulations have assumed a fixed central potential, which is not allowed to move in response to non-axisymmetric structures in the disk, the $m = 1$ Fourier component should not be considered to be an accurate measurement for all simulations considered in this chapter. is either not plotted or not considered. Fixing the central star could, in principle, suppress the growth of a $m = 1$ mode via SLING amplification (see section \ref{sec:ind:SMm1} for a more detailed discussion of the SLING mechanism.) However, simulations considered in chapters \ref{chap:indirect} and \ref{chap:planet} have included the indirect potential, which takes into account the motion of the central star. In these cases, the $m = 1$ modes are accurately reflected in all the analyses. Frequently, the $A_m$ values are computed for many snapshots of disk evolution and then time-averaged so as to remove fluctuations on small temporal scales. When referring to $A_m$ values, it is understood that I am referring to the value of a snapshot, while $\left< A_m \right>$ refers to the time-averaged quantity. Another relevant quantity is the sum of the time-averaged values over all $m$-values, which is denoted as $\left<A_\Sigma\right>$.

\subsection{Periodicity of Coherent Modes}\label{sec:IC:periodogram}

Another powerful tool in examining structure caused by GI activity is the periodogram. As in the $A_m$ analysis, I Fourier decompose the non-axisymmetric density fluctuations locally in $\cos(m\phi + \phi_m)$. In this case $\phi_m$ contains the phase information for a $m$-armed structure. Only the midplane density structures are decomposed giving the Fourier components as a function of $\varpi$. After constructing a time series of these Fourier components from the density files saved throughout a simulation, I search for any pattern frequencies by constructing a Lomb-Scargle periodogram \citep{scargle1982,horne1986} of $\cos(m\phi)$ at each radius. For each $m$-value, I then construct a map of the periodic structures in the phase angle as a function of radius and frequency. When a period exhibits a large power relative to the background and spans a large radial range in this plot, it indicates the presence of global GI modes or waves. I also identify the corotation radii and inner and outer Linblad resonances associated with each of these $m$-values. Typically I investigate the periodicity of low-order modes only, as higher-order modes tend to be plagued by small scale noise.

\subsection{Mass Transport in Protoplanetary Disks}\label{sec:IC:masstransport}

In order for mass transport to occur in a protoplanetary disk, material must either lose angular momentum, and be accreted inward, or gain angular momentum, and be accelerated outward. The transport of angular momentum must be facilitated by some form of torque. \citet{hartmann1998a} provides a review of several possible mechanisms for angular momentum transport in accretion disks. For physical parameters in protoplanetary disks, molecular viscosity is far too small to play a significant role in mass accretion. Another possibility that has been considered is turbulent convection. However, several detailed studies \citep{ryu1992,cabot1992,stone1996} indicate that this mechanism will act to transport angular momentum inward, resulting in outward mass transport. Another possible mechanism is the magnetorotational instability (MRI) \citep{balbus1997}, which acts to drive magnetohydrodynamic (MHD) turbulence. The MRI is a robust instability that can develop in disks with a vanishingly small initial magnetic field.

The basic idea behind the instability is as follows. Initially a magnetic field line radially connects two adjacent annuli. If the disk has a Keplerian rotation profile, then the inner annulus will have a larger angular velocity than the outer annulus. This causes the field lines to stretch, which is opposed by the magnetic field. This, in turn, will accelerate the outer annulus with respect to the inner annulus as the field lines attempt to return to equilibrium. As angular momentum is transferred outward, the inner annulus will move further inward and the outer annulus will move further outward exacerbating the imbalance. As the field lines are stretched further and material is transferred over an increasing radial extent, turbulent motions develop \citep{hawley1995}. Even with a vanishing small magnetic field, the instability will grow because the initial stretching of the magnetic field amplifies the field, i.e., the instability exhibits a dynamo effect \citep{hawley1996}.

However, there are a few constraints on the system in order for the MRI to be active and effective. Because the instability is driven by the shearing of the magnetic field, the field cannot be too strong. If the field is strong enough to enforce the co-rotation of neighboring annuli, then the field lines will not be stretched and the instability will not develop. Additionally, the field must be able to couple effectively to the gas disk. The field itself interacts with charged particles, which are plentiful in many astrophysical settings. However, in a cold disk, shielded from external irradiation by dust grains, the number of ions in the gas can become very small. This can lead to {\it dead zones} in the disk, areas where the magnetic field cannot effectively couple to the disk and therefore the MRI cannot grow. The MRI may be active near the surface of the disk where irradiation can keep the ionization fraction large enough. In such cases, disks may undergo layered accretion where MRI-driven turbulence causes accretion near the disk surface but there is no accretion in the dead zone near the disk midplane \citep{gammie1996}. 

Much work has been done to characterize the magnitude of angular momentum transport caused by the MRI in protoplanetary disks. Early works by \citet{hawley1995} and \citet{brandenburg1995} indicated that the effective $\alpha$ \citep{shakura1973} produced by the MRI could range from 0.001 to 0.1 depending on the particulars of the field geometry. However, recent studies have shown the importance of numerical dissipation \citep{fromang2007a}, viscous and resistive dissipation \citep{lesur2007}, and field geometry \citep{simon2009}. At present these quantities are not well known for protoplanetary disk environments, and so the effective $\alpha$ produced by the MRI in protoplanetary disks in locations where it can act is poorly quantified.

Gravitational instabilities can also facilitate the transport of angular momentum. For Toomre $Q \lesssim 1.7$ \citep{durisen2007}, gravitational interactions produce spiral waves. If these spiral waves are trailing waves and can be sustained by the instability despite the effects of shear, then a concentration of mass will build in a trailing spiral arm. The leading and trailing portions of the arm will be gravitationally attracted to each other causing the trailing portion (or outer radii) to experience a positive torque. As in the MRI, attraction between inner and outer annuli attempts to force the disk into solid body rotation and transfers angular momentum outward. The chief difference is that the MRI is driven by magnetic attraction while GIs are driven by gravitational attraction.

Another key difference is the conditions under which each of the instabilities will develop. As discussed above, the MRI is a fairly robust instability that will develop in the presence of a vanishingly small magnetic field given that the disk is able to couple to the field effectively. On the other hand, GIs require a Toomre $Q \lesssim 1.7$ to develop. As outlined in \S \ref{sec:Intro:disktheory}, three factors in the disk determine the Toomre $Q$ parameter. The local pressure and rotation act to stabilize the disk against short and long wavelength instabilities, respectively. The destabilizing force is gravity. Therefore, the conditions are right for GIs where there is rapid cooling, large mass concentrations, and little rotational support. Typically, this occurs in the cold outer regions of a massive protoplanetary disk. 

If the condition $Q \lesssim 1.7$ is satisfied, this does not in itself mean that GIs will be effective in transporting angular momentum. The two major characteristics that determine the effectiveness of angular momentum transport, or the magnitude of the gravitational torque, are the strength of the GIs and the coherence of the structures produced by the instability. These can be measured by the analyses described in the preceding sections. The strength of the GIs can be quantified by looking at the $A_m$ values because it is a measure of the mass in non-axisymmetric $m$-armed structures. Additionally, one can look at $A_\Sigma$ as a measure of the total strength of GIs for all $m$-values in the disk.

The other factor that determines how effectively GIs can transport angular momentum is the coherence of the patterns produced by the instability. It is possible, for example, for a given $m$-value to have a large measured $A_m$ but, since the $A_m$ measure does not take phase information into account, the power could be phased differently at different radii. In an extreme case, if the power at different radii were phased properly, the net torque could be negligible over a large radial extent in the disk. To determine the coherence of structures, I use the periodogram, described in \S\ref{sec:IC:periodogram}, to look for stripes of power that have the same pattern speed over a large range of radii. In addition to determining their pattern speed, I can determine where they are most effective in transporting angular momentum by looking at the corotation and inner and outer Linblad resonances of these structures. It should be noted that the $A_m$ method for determining the strength of GIs and the periodogram method for determining coherence are independent, that is, the $A_m$ measures only the amplitude and has no information on the coherence while the periodogram measures only the coherence and has no information of the amplitude of the pattern. In principle, it would be possible to have a strong incoherent pattern {\it and} a weak coherent pattern that could be misinterpreted as a single strong coherent pattern. However, this is unlikely and in all cases I look at other time varying quantities that link these two measures, such as net torques.

\subsection{Gravitational Torques from Global Instabilities}\label{sec:IC:torque}

A coherent structure with a large density enhancement will produce a gravitational torque and, in the case of a trailing spiral, cause angular momentum to be transferred outward. As in \citet{lyndenbell1972}, by taking a cylindrical surface at some radius $\boldsymbol{r}$, one can calculate the torque couple $C$ of the portion of the disk at radii greater than $\boldsymbol{r}$ on the portion of the disk inside $\boldsymbol{r}$ by integrating the stress tensor $T$ over the surface of the cylinder,

\begin{equation}
C = \int \boldsymbol{r} \times T \cdot \dif S.
\end{equation}
In a protoplanetary disk, the chief contributions to the stress tensor, in the absence of a magnetic field, come from the Reynolds stress and gravitational stress. The relevant component of the Reynolds stress tensor is
\begin{equation}
T^{Reyn}_{\varpi\phi} = \rho \delta v_{\varpi} \delta v_{\phi}
\end{equation}
where $\rho$ is the gas density, $\delta v_{\varpi}$ is the velocity fluctuation of the fluid from the mean flow in the $\varpi$-direction, and $\delta v_{\phi}$ is the velocity fluctuation of the fluid from the mean flow in the $\phi$-direction. The torque that arises from this stress can then be computed by
\begin{equation}
C^{Reyn}_z = \int_0^{2\pi}\!\!\int_{-\infty}^{\infty}\!T^{Reyn}_{\varpi\phi} \varpi^2 \,\dif \phi \,\dif z.
\end{equation}

However, in order to determine the components of the Reynolds stress accurately, one must first define a mean flow and measure the fluctuations about that flow. In a GI active disk dominated by non-linear gravitational interactions, defining a mean flow can be difficult. Although other groups have found the Reynolds stresses to be comparable to the gravitational stresses \citep{gammie2001}, in the instances where our group and others have attempted to quantify the Reynolds stress, it has been found to be relatively small in comparison with the gravitational stress in GI active regions \citep{lodato2004,boley2006,boleyphd2007}. For this reason, I will focus on the torque due to the gravitational stress. 

Instead of integrating the gravitational stress tensor over the cylindrical surface, one may replace the surface integral with a volume integral
\begin{equation}
C=\int \rho \boldsymbol{r}\times \nabla\Phi \dif V,
\end{equation}
where $\Phi$ is the gravitational potential. As in \citet{boley2006}, I am only interested in the torque about the $z$-axis. So,
\begin{equation}
C_{z} = \int_V \rho\frac{\partial\Phi}{\partial\phi} \dif V.
\label{eq:IC:gravtorquez}
\end{equation}
This allows me to compute the torque of the outer disk on the inner disk as a function of $\varpi$ by integrating over the entire volume enclosed by a cylinder of radius $\varpi$. Each value of $\varpi$ represents the computation with new cylindrical radius, yielding a plot of the torque versus $\varpi$.

In addition, I use the Fourier coefficients $a_m$ and $b_m$ from equation \eqref{eq:IC:fourier} to construct a density distribution with only $m$-armed structure. The density can be reconstructed from the Fourier coefficients $c_n$ and $s_n$ given by,
\begin{subequations}
\begin{align}
c_n &= \frac{1}{\pi} \int_{-\pi}^\pi \rho \cos\left(n\phi \right) \dif \phi, \qquad n \geq 0\\
\intertext{and}
s_n &= \frac{1}{\pi} \int_{-\pi}^\pi \rho \sin\left(n\phi \right) \dif \phi, \qquad n \geq 1
\end{align}
\end{subequations}
by combining them so that
\begin{equation}
\rho = \frac{c_0}{2} + \sum_{n=1}^{N} \left[ c_n \cos \left( n \phi \right) + s_n \sin \left( n \phi \right) \right].
\end{equation}
I can then construct the density from a single $m$-valued by
\begin{equation}
\rho_n = c_n \cos(n\phi) + s_n \sin(n\phi).
\end{equation}
Once these density maps have been created I compute the torque due to each of the $m$-armed structures and can thereby measure the relative importance of each of the modes in transporting angular momentum. 

\subsection{Effective $\alpha$}\label{sec:IC:alphadisk}

\citet{shakura1973} first parametrized the viscosity in a steady-state disk with the $\alpha$ parameter. Although their original work applied to X-ray bright accretion disks around black holes, their approach has proved fruitful in many astrophysical situations. In the $\alpha$-disk formalism, the effective kinematic viscosity, $\nu$, can be written as
\begin{equation}
\nu = \alpha c_s H,
\end{equation}
where $c_s$ is the sound speed and $H$ is the disk scale height. With this formalism, when the accretion rate $\dot{M}$ is constant in both time and radius, $\dot{M} = 3\pi\nu\Sigma = 3\pi \alpha c_s^2 \Sigma / \Omega$, where $\Sigma$ the surface density and $\Omega$ the rotational frequency. The steady-state $\alpha$-disk formalism makes no assumption about the nature of the angular momentum transport, except that it resembles viscosity and hence, in a steady-state, and cooling balance locally. \citet{gammie2001} envisioned a razor thin gravitationally unstable disk where the heating due to local turbulent dissipation caused by GIs was balanced by the local cooling rate. Assuming a local cooling time $t_{cool} $, \citeauthor{gammie2001} found that
\begin{equation}
\alpha = [\Gamma (\Gamma -1)\frac{9}{4} \Omega t_{cool}]^{-1},
\label{eq:IC:gammiealpha}
\end{equation}
where $\Gamma$ is the two-dimensional adiabatic index. The two-dimensional index relates to the three dimensional index $\gamma$ by $\Gamma = (3\gamma -1)/(\gamma +1)$ for a non-self-gravitating disk and $\Gamma = 3 - 2/\gamma$ for a strongly self-gravitating disk.

A key question in the evolution of GI active protoplanetary disks is whether the GIs dissipate energy locally (i.e. as $t_{cool}\Omega = \mathrm{const.}$) or globally (i.e. as $t_{cool} = \mathrm{const.}$). As in \citet{lodato2004}, to compare the effective $\alpha$ in my simulations to \citeauthor{gammie2001}'s prediction, I compute $\alpha$ from stresses in the disk,
\begin{equation}
\alpha (\varpi ) = \left | \frac{\dif \ln \Omega}{\dif \ln \varpi} \right |^{-1} \frac{ \mathcal{T}^{grav}_{\varpi \phi} + \mathcal{T}^{Reyn}_{\varpi \phi} }{\Sigma c^2_s}.
\label{eq:IC:alpha}
\end{equation}
Here $\Omega$ is the azimuthally averaged rotation speed, $\mathcal{T}$ the vertically integrated, azimuthally averaged stress tensor, $\Sigma$ the surface density, and $c_s$ the azimuthally averaged midplane sound speed. Due to the fact that most of the disks I consider are substantially extended in the $z$ direction, it is unclear whether using the midplane sound speed with vertically integrated quantities, such as $\Sigma$, will produce accurate results. By testing several possible ways of determining the product, $\Sigma c_s^2$, I found that using $\int_z \rho c_s^2 dz$ yielded the best results. Furthermore, due to numerical uncertainties in the determination of the Reynolds stress and its small magnitude compared to the gravitational stress, I neglect the Reynolds stress. The gravitational torque from equation \eqref{eq:IC:gravtorquez} can be used to determine the vertically integrated and azimuthally averaged gravitational stress tensor by 
\begin{eqnarray}\label{eq:IC:tgrav}
\mathcal{T}^{grav}_{\varpi \phi} (\varpi) &=& \frac{1}{2 \pi \varpi^2}C^{grav}_z\\
&=&-\frac{1}{2 \pi \varpi^2} \int_{V(\varpi)}\rho\frac{\partial\Phi}{\partial\phi}\dif V. \nonumber
\end{eqnarray}

One should note, however, that this analysis produces an {\it effective} $\alpha$ caused by gravitational stresses. Strictly speaking, a gravitationally unstable disk is not in a steady state with $\dot{M}$ constant in both time and radius. Calculating the effective $\alpha$ allows one to measure the efficacy of the angular momentum transport. Where most MRI simulations predict an $\alpha \gtrsim 0.01$, with the precise value depending on the numerical resolution and assumptions about dissipation, stratification and ionization \citep{fromang2007a, guan2009, davis2009,hartmann1998b}.

\section{The Simulations}\label{sec:IC:initialmodel}

All the models considered in this chapter have a disk to star mass ratio of $\approx 0.14-0.15$.  The initial models were generated using a grid-based self-consistent field (SCF) scheme \citep{hachisu1986,pickett1996,pickett2003} that is able to generate a disk with an arbitrary power law for the surface density, $\Sigma = \Sigma_0 r^{-p}$. 

The volumetric cooling rate is constant everywhere in the disk and is defined by the parameter
\begin{equation}\label{eq:IC:tcool}
t_{cool}=\epsilon/\Lambda= \mathrm{const},
\end{equation}
where $\epsilon$ is the internal energy density and $\Lambda$ is the volumetric cooling rate. Here $t_{cool}$ is given in outer rotation periods (ORPs) where 1 ORP ($\approx$ 250 years) is defined as the rotation period at radial zone 200 ($\approx 33$AU) of the G1.7TC2P0.5 disk, see table \ref{tbl:IC:initcond}, chosen arbitrarily \citep{mejia2005}. Although the assumption of $t_{cool}=\mathrm{const.}$ is not particularly realistic, it allows for a much more extensive search of parameter space because the computations are less expensive than those with more realistic radiative physics. Because of variations in the initial models, the value of the ORP in physical units varies. However, the variation is no more than 25\%.

Each of the disks have a $1.0 M_{\odot}$ central star, which is either treated as a point mass or a mass distribution and fixed to the center of the grid. The disks initially range from between 1.8 -- 2.3 AU to 40 AU in extent. For chapter \ref{chap:compare}, all disks have an initial resolution of ($r,\phi,z$) = (256,128,32) which gives $\approx 1/6$ AU cell widths in the r and z direction. Each of the disks has $1.4\leq \mathrm{Q_{min}}\leq 2.0$ making them marginally unstable initially. 

Overall, I studied eight different initial disk models, which I will refer to by their $\gamma$, $t_{cool}$, and initial surface density profile. (e.g. G1.7TC2P1 is a disk with $\gamma = 1.7$, $t_{cool} = 2$, and $\Sigma = \Sigma_0 r^{-1}$.) The axisymmetric disk models are all seeded with an initial random cell-to-cell perturbation. Typically the amplitude of this perturbation is small, most models have a density perturbation amplitude of $\xi = 0.0001$ (see \S\ref{sec:IC:randpert}). The RAND indicator at the end of a run name identifies those runs which did not use the standard $\xi = 0.0001$ initial perturbation but instead used some larger amplitude. The number following RAND represents the amplitude of initial density perturbation used to seed the GIs. For those simulations with a number, the number represents the percent perturbation (e.g. RAND5 means $\xi = 0.05$).

\begin{table}
\renewcommand{\arraystretch}{1.25}
  \begin{tabular*}{1.0\textwidth}{@{\extracolsep{\fill}}cp{1.2cm}p{1cm}cc}
  \hline
Model&$p$&\centering ORP (yr) & $M_{\mathrm{disk}}/M_{\ast}$ & $\mathrm{l_{max}}$\\
  \hline\hline  
  G1.7TC2P0.5 & 0.445 & \centering  253 & 0.141 & 128\\
  G1.7TC2P1 & 0.931 & \centering 303 & 0.153 &64,128,256,512\\
  G1.7TC2P1.5 & 1.485 & \centering 242 & 0.141 &128\\
  G1.7TC0.6P1 & 0.931 & \centering 303 & 0.153 &512\\
  G1.4TC1P1 & 0.931 & \centering 303 & 0.153 & 512\\
  G1.4TC3P1 & 0.931 & \centering 303 & 0.153 & 256\\
  G1.7TC2P1RAND1 & 0.931 & \centering 303 & 0.153 & 128\\
  G1.7TC2P1RAND5 & 0.931 & \centering 303 & 0.153 & 128\\
 \hline
\end{tabular*}
\vspace{0.1in}
\caption[Disk initial conditions]{Initial conditions for each of the disk simulations discussed in this chapter.}
\label{tbl:IC:initcond}
\end{table} 
\renewcommand{\arraystretch}{1}

\section{Variation of Initial Surface Density Profile}\label{sec:IC:surfden}

To study the effect of the initial surface density profile on the onset and outcome of GIs, I analyzed three disk simulations G1.7TC2P0.5, G1.7TC2P1, and G1.7TC2P1.5. All three disks use the volumetric cooling in equation \eqref{eq:IC:tcool} with $t_{cool} = 2$ ORPs. The SCF scheme described by \citet{pickett1996} was used to generate equilibrium models with surface density profiles that followed $\Sigma \sim r^{-p}$ for $p$ of $1/2$, $1$, and $3/2$. Due to numerical inaccuracies in the determination of the equilibrium state, the surface density profiles produced were not exact. The actual surface density profile produced by the SCF scheme was measured by using a linear least squares fit for $p$. Those values can be found in Table \ref{tbl:IC:initcond}. The numerical inaccuracies in the SCF code also introduced a slight variation in  the disk to star mass ratio between models. Additionally, these variations cause a small difference in the physical value of the ORP. All of these variations are minor compared to the drastic difference in initial surface density profile. 

\begin{sidewaysfigure}
\begin{center}
\unitlength1in
\subfigure{\includegraphics[width=0.23\textwidth]{figures/chap3/p5den/P.5.0.eps}}
\subfigure{\includegraphics[width=0.23\textwidth]{figures/chap3/p5den/P.5.0a.eps}}
\subfigure{\includegraphics[width=0.23\textwidth]{figures/chap3/p5den/P.5.1.eps}}
\subfigure{\includegraphics[width=0.23\textwidth]{figures/chap3/p5den/P.5.2.eps}}\\
\subfigure{\includegraphics[width=0.23\textwidth]{figures/chap3/p5den/P.5.3.eps}}
\subfigure{\includegraphics[width=0.23\textwidth]{figures/chap3/p5den/P.5.4.eps}}
\subfigure{\includegraphics[width=0.23\textwidth]{figures/chap3/p5den/P.5.4a.eps}}
\subfigure{\includegraphics[width=0.23\textwidth]{figures/chap3/p5den/P.5.5.eps}}\\
\subfigure{\includegraphics[width=0.23\textwidth]{figures/chap3/p5den/P.5.6.eps}}
\subfigure{\includegraphics[width=0.23\textwidth]{figures/chap3/p5den/P.5.7.eps}}
\subfigure{\includegraphics[width=0.23\textwidth]{figures/chap3/p5den/P.5.8.eps}}
\subfigure{\includegraphics[width=0.23\textwidth]{figures/chap3/p5den/P.5.9.eps}}
\caption[Time series of midplane and meridional densities for G1.7TC2P0.5 disk]{Midplane and meridional densities in logarithmic scale for several times in the G1.7TC2P0.5 simulation. The axes have units of AU and the time is given in ORPs in the upper right of each panel. The series starts at $T = 0$ ORPs and proceeds left to right and top to bottom to the end of the simulation at $\approx 18$ ORPs.}
\label{fig:IC:p5multiden}

\end{center}
\end{sidewaysfigure}

\begin{sidewaysfigure}
\begin{center}
\unitlength1in
\subfigure{\includegraphics[width=0.23\textwidth]{figures/chap3/p1den/P1.0.eps}}
\subfigure{\includegraphics[width=0.23\textwidth]{figures/chap3/p1den/P1.1.eps}}
\subfigure{\includegraphics[width=0.23\textwidth]{figures/chap3/p1den/P1.2.eps}}
\subfigure{\includegraphics[width=0.23\textwidth]{figures/chap3/p1den/P1.3.eps}}\\
\subfigure{\includegraphics[width=0.23\textwidth]{figures/chap3/p1den/P1.4.eps}}
\subfigure{\includegraphics[width=0.23\textwidth]{figures/chap3/p1den/P1.5.eps}}
\subfigure{\includegraphics[width=0.23\textwidth]{figures/chap3/p1den/P1.6.eps}}
\subfigure{\includegraphics[width=0.23\textwidth]{figures/chap3/p1den/P1.7.eps}}\\
\subfigure{\includegraphics[width=0.23\textwidth]{figures/chap3/p1den/P1.8.eps}}
\subfigure{\includegraphics[width=0.23\textwidth]{figures/chap3/p1den/P1.9.eps}}
\subfigure{\includegraphics[width=0.23\textwidth]{figures/chap3/p1den/P1.10.eps}}
\subfigure{\includegraphics[width=0.23\textwidth]{figures/chap3/p1den/P1.11.eps}}

\caption[Time series of midplane and meridional densities for G1.7TC2P1 disk]{Midplane and meridional densities in logarithmic scale for several times in the G1.7TC2P1 simulation. The axes have units of AU and the time is given in ORPs in the upper right of each panel. The series starts at $T = 0$ ORPs and proceeds left to right and top to bottom to the end of the simulation at $\approx 18$ ORPs.}
\label{fig:IC:p1multiden}

\end{center}
\end{sidewaysfigure}

\begin{sidewaysfigure}
\begin{center}
\unitlength1in
\subfigure{\includegraphics[width=0.23\textwidth]{figures/chap3/p15den/P1.5.0.eps}}
\subfigure{\includegraphics[width=0.23\textwidth]{figures/chap3/p15den/P1.5.1.eps}}
\subfigure{\includegraphics[width=0.23\textwidth]{figures/chap3/p15den/P1.5.2.eps}}
\subfigure{\includegraphics[width=0.23\textwidth]{figures/chap3/p15den/P1.5.3.eps}}\\
\subfigure{\includegraphics[width=0.23\textwidth]{figures/chap3/p15den/P1.5.4.eps}}
\subfigure{\includegraphics[width=0.23\textwidth]{figures/chap3/p15den/P1.5.5.eps}}
\subfigure{\includegraphics[width=0.23\textwidth]{figures/chap3/p15den/P1.5.6.eps}}
\subfigure{\includegraphics[width=0.23\textwidth]{figures/chap3/p15den/P1.5.7.eps}}\\
\subfigure{\includegraphics[width=0.23\textwidth]{figures/chap3/p15den/P1.5.8.eps}}
\subfigure{\includegraphics[width=0.23\textwidth]{figures/chap3/p15den/P1.5.9.eps}}
\subfigure{\includegraphics[width=0.23\textwidth]{figures/chap3/p15den/P1.5.10.eps}}
\subfigure{\includegraphics[width=0.23\textwidth]{figures/chap3/p15den/P1.5.11.eps}}

\caption[Time series of midplane and meridional densities for G1.7TC2P1.5 disk]{Midplane and meridional densities in logarithmic scale for several times in the G1.7TC2P1.5 simulation. The axes have units of AU and the time is given in ORPs in the upper right of each panel. The series starts at $T = 0$ ORPs and proceeds left to right and top to bottom to the end of the simulation at $\approx 18$ ORPs.}
\label{fig:IC:p15multiden}

\end{center}
\end{sidewaysfigure}

\subsection{Overall Evolution}

As in \citet{mejia2005}, the disks begin in an axisymmetric phase, experience a burst of spiral activity, and transition to an asymptotic phase where the disk cooling is roughly balanced by heating due to shocks and gravitational contraction. In this last phase, average quantities are changing slowly, even though the detailed structure can change on a dynamic timescale. Mass transport is generally the strongest during the onset of the gravitational instabilities. During this initial burst of activity, the overall disk structure can change dramatically. Although all three of the disks seem to follow this general trend, there are some significant variations. Figures \ref{fig:IC:p5multiden}, \ref{fig:IC:p1multiden}, and \ref{fig:IC:p15multiden} show several snapshots of the midplane and a meridional cut of the density for the three simulations. The times given in each panel are in ORPs and range from the initial axisymmetric disk to the end of the simulations around 18 ORPs. The disk cools in the axisymmetric phase from 0 -- 2 ORPs for the G1.7TC2P0.5 and G1.7TC2P1 disks. This is followed by a burst of GI activity initiated in the low $Q$ region, seen in the next three panels from 2 -- 6 ORPs. The burst transitions from strong GI activity to a quasi-steady state with 1.5 $\lesssim Q \lesssim$ 2.0. The transition phase occurs from 6 -- 12 ORPs and is followed by the asymptotic phase, which lasts through the end of the simulation. The G1.7TC2P1.5 disk displays much different behavior. It remains nearly axisymmetric until 4 ORPs, and then undergoes a burst from 4 -- 9 ORPs.

A similar behavior can also be seen in the Fourier amplitudes. Figure \ref{fig:IC:amt} shows $A_m$ as a function of time for each of the disks through the burst phase and into the asymptotic phase. The amplitudes plotted in figure \ref{fig:IC:amt} show only contributions from radii greater than 25 AU due to $m = 1$ signal contamination at small radii. One can see that for the G1.7TC2P0.5 and G1.7TC2P1 disks the amplitudes begin to grow exponentially near 2 ORPs and reach a peak amplitude between 5 and 6 ORPs; this is the burst phase. Density structures for $m = 1-5$ are shown for the burst onset in the G1.7TC2P1 disk in figure \ref{fig:IC:mdecomp}. The decomposed density structures are shown with an additional axisymmetric background to avoid display issues with the logarithmic scale. One can see in both the $A_m$ plots (figure \ref{fig:IC:amt}) and the density decompositions (figure \ref{fig:IC:mdecomp}) that strong $m = 1-5$ patterns are present in the burst. After the burst phase, the amplitudes remain high through the asymptotic phase. 

\begin{figure}[p]
\centering
\unitlength1in
\begin{minipage}[t]{6.3in}
\centering
\includegraphics[scale=.5]{figures/chap3/amgrowthP0.5_25AU.eps}
\end{minipage}\\
\begin{minipage}[t]{6.3in}
\centering
\includegraphics[scale=.5]{figures/chap3/amgrowthP1_25AU.eps}
\end{minipage}
\begin{minipage}[t]{6.3in}
\centering
\includegraphics[scale=.5]{figures/chap3/amgrowthP1.5_25AU.eps}
\end{minipage}
\caption[Fourier components as a function of time for different initial surface density profiles]{Each panel shows the Fourier components $A_m$ as a function of time for $m = 1 - 6$. The G1.7TC2P0.5, G1.7TC2P1, and G1.7TC2P1.5 simulations are shown from top to bottom. The amplitudes are shown from the initial models through the asymptotic phase. Contribution to the $A_m$ components is only calculated for $\varpi > 25$ AU due to contamination from a spurious $m = 1$ signal. Note the protracted burst, which occurs later, and the dominance of the $m = 5$ and $6$ modes in the G1.7TC2P1.5 simulation.}
\label{fig:IC:amt}
\end{figure}

\begin{figure}[p]
\unitlength1in
\begin{minipage}{0.5\linewidth}
\centering
\includegraphics[width=2.5in]{figures/chap3/density.m0.eps}
\end{minipage}
\hspace{0.25in}
\begin{minipage}{0.5\linewidth}
\centering
\includegraphics[width=2.5in]{figures/chap3/density.m1.eps}
\end{minipage}
\vspace{0.25in}

\begin{minipage}{0.5\linewidth}
\centering
\includegraphics[width=2.5in]{figures/chap3/density.m2.eps}
\end{minipage}
\hspace{0.25in}
\begin{minipage}{0.5\linewidth}
\centering
\includegraphics[width=2.5in]{figures/chap3/density.m3.eps}
\end{minipage}
\vspace{0.25in}

\begin{minipage}{0.5\linewidth}
\centering
\includegraphics[width=2.5in]{figures/chap3/density.m4.eps}
\end{minipage}
\hspace{0.25in}
\begin{minipage}{0.5\linewidth}
\centering
\includegraphics[width=2.5in]{figures/chap3/density.m5.eps}
\end{minipage}
\caption[Fourier reconstructed midplane and meridional densities]{Panels showing the midplane and meridional densities of the Fourier decomposed $m$-values for the G1.7TC2P1 disk during the burst phase at $T = 5.5$ ORPs. The top left panel is the full density structure. Proceeding left to right and top to bottom $m = 1$, $2$, $3$, $4$, and $5$ are shown superimposed on the underlying axisymmetric structure, $m = 0$. Densities are given in code units using a logarithmic scale.}
\label{fig:IC:mdecomp}
\end{figure}

The G1.7TC2P1.5 disk shows very different behavior. Here, the amplitude growth does not begin until somewhere around 4 ORPs, and the peak amplitude is reached between 8 and 9 ORPs. Since the amplitudes grow more slowly in the G1.7TC2P1.5 disk, the burst is less pronounced, the disk does not expand as much, and less mass is transported through the disk (see figure \ref{fig:IC:mdots}).

\subsection{The Burst Phase}

The burst phase is clearly seen in the $T = 4-6$ ORPs panels of figures \ref{fig:IC:p5multiden} and \ref{fig:IC:p1multiden} and in the $T = 6-8$ ORPs panels of figure \ref{fig:IC:p15multiden}. These figures also illustrate the dominate $m$-armed spirals in the burst phase, $m=4$ and 5 for G1.7TC2P0.5 and G1.7TC2P1 and $m=5$ and 6 for G1.7TC2P1.5. The burst phase and dominant modes can also be seen in the steep rise in $A_m$ in figure \ref{fig:IC:amt}. The burst can be defined by the time period between the onset of linear growth of the low-order modes and the peak of the growth as measured by eye from figure \ref{fig:IC:amt}. Using this definition, the burst was determined to be from 2.1 to 5.5 ORPs for the G.17TC2P0.5 disk, from 2.0 to 6.0 ORPs for the G1.7TC2P1 disk, and from 4.0 to 8.8 ORPs for the G1.7TC2P1.5 disk. The shift in dominant mode is due to the shortening of the most unstable wavelength as the surface density decreases at the radius of the minimum $Q$. The most unstable wavelength is given by 
\begin{equation}
\lambda = \frac{2\pi^2 G \Sigma}{\kappa^2}
\end{equation}
 \citep{toomre1981}, here $\kappa$ is determined primarily by the mass of central star, which remains constant. Since the instabilities initiate at radii near the minimum Toomre $Q$, which are the same for all three disks, and the G1.7TC2P0.5 disk has a significantly larger $\Sigma$ at these radii, one would expect the most unstable wavelength to be longer for the G1.7TC2P0.5 disk. This, in turn, should result in lower order $m$-values being dominant when the instabilities are initiated, which is what is seen in figure \ref{fig:IC:amt}. 

Table \ref{tbl:IC:qmins} gives the minimum Toomre $Q$ value near the time when the amplitudes begin to grow, and the radius at which $Q$ is smallest. The minimum $Q$ tends to be at the same radius for all three disks, due to the fact that the initial $Q$ profile is similar for all the disks. Another factor is the global cooling time which tends to bias the cooling to the outer disk, i.e. $t_{cool} =2$ represents a cooling time of two rotational periods at 33 AU and 5 rotational periods at 11 AU. 

Because all three disks have similar minimum $Q$ values at similar radii, the key difference is the amount of mass involved in this initial burst. For the G1.7TC2P1.5 case, there is less disk mass at radii $\gtrsim 30$ AU than in the G1.7TC2P1 and G1.7TC2P0.5 disks. The protracted burst seen in the G1.7TC2P1.5 disk is most likely due to the fact that there is less mass in the region where the disk first becomes unstable. In fact, the G1.7TC2P1.5 disk appears to burst in radial stages as opposed to globally as in the G1.7TC2P1 and G1.7TC2P0.5 disks. In the G1.7TC2P1 and G1.7TC2P0.5 disks, $Q$ drops to its minimum and then GIs erupt strongly heating the disk over a range of radii (i.e. 15 AU to 40 AU). In the G1.7TC2P1.5 case the heating is much more localized to the area around the $Q$ minimum. Regions of the disk interior to the $Q$ minimum continue to cool until they become unstable at some later time. Due to the lack of mass in the outer disk the initial instability in the G1.7TC2P1.5 disk is too weak to cause shock heating over a large radial range.  
\begin{table}
\renewcommand{\arraystretch}{1.25}
\begin{tabular*}{1.0\textwidth}{@{\extracolsep{\fill}}cccc}
\hline
Model&$Q_{min}$&$Q_{min}$ Radius (AU)&Time (ORPs)\\
\hline\hline  
G1.7TC2P0.5&1.0&33&2.1\\ 
G1.7TC2P1&1.3&34&1.9\\ 
G1.7TC2P1.5&1.1&33&4.2\\ 
\hline
\end{tabular*}
\vspace{0.1in}
\caption[Minimum $Q$ values for different initial surface density profiles]{Minimum $Q$ values and their positions near the time when non-axisymmetric amplitudes begin to grow.}
\label{tbl:IC:qmins}
\end{table} 
\renewcommand{\arraystretch}{1}

One of the consequences of weaker GI activity and higher-order modes playing more of a role in the burst is that the mass transport rates differ dramatically for the different $\Sigma(\varpi)$. As figure \ref{fig:IC:mdots} shows, the peak mass transport rate in the G.17TC2P0.5 disk is about a factor of five greater than the peak mass transport rate in the G1.7TC2P1.5 disk. Here the burst is defined by the time period determined by eye from figure \ref{fig:IC:amt}, as explained earlier in this section. Another interesting fact to note from figure \ref{fig:IC:mdots} is that the integrated mass transport is much larger for the G.17TC2P0.5 disk since mass is being transported over a larger radial range. 

\begin{figure}[p]
\centering
\includegraphics[scale=0.9]{figures/chap3/mdotburstsigma.eps}
\caption[Burst mass accretion rates for different initial surface density profiles]{The radial mass accretion rates are shown in $M_\odot/yr$ for the G1.7TC2P0.5 (black), G1.7TC2P1 (red), and G1.7TC2P1.5 (blue). These rates are measured during the burst phase of each of the simulations. The time period encompassing the burst is different for each of the simulations. The burst is measured from $2.1-5.5$ ORPs for the G1.7TC2P0.5 disk, $2.0-6.0$ ORPs for the G1.7TC2P1 disk, and $4.0-8.8$ ORPs for the G1.7TC2P1.5 disk.}
\label{fig:IC:mdots}
\end{figure}

\subsection{The Asymptotic Phase}

In order to predict the outcome of the reordering of the disk due to the burst phase, consider a disk in the asymptotic phase where heating and cooling are roughly balanced. This state, termed ``gravitoturbulence" by \citet{gammie2001}, forces the Toomre $Q$ parameter to a marginally unstable value which is typically constant throughout the GI active region (see figure \ref{fig:IC:constq}). \citet{boley2008} showed that for an axisymmetric disk with constant $\gamma$, a Keplerian rotation profile, and negligible self-gravity,
\begin{eqnarray*}
\Sigma(\varpi) &=& \pi^{-\left(3\gamma+1\right)/4}\left(\frac{2}{\gamma-1}\left\{\frac{\Gamma\left[\gamma/\left(\gamma-1\right)\right]}{\Gamma\left[\left(3\gamma-1\right)/\left(2\gamma-1\right)\right]}\right\}^2\right)^{\left(1-\gamma\right)/4}\\
&&\times[\gamma K(\varpi)]^{1/2}[GQ(\varpi)]^{-(\gamma+1)/2}\Omega (\varpi)^\gamma.
\label{eq:IC:sigmaprof}
\end{eqnarray*}
Here $\gamma$ is the polytropic constant and $K$ is the entropy constant in the equation of state (i.e. $P = K\rho^\gamma$). With roughly constant $Q$ and $K$, as in figure \ref{fig:IC:constq}, equation \eqref{eq:IC:sigmaprof} predicts that $p = 5/2$. When a least squares fit for $p$ is performed on the final surface density profiles, I find that $p = 2.3, 2.7,\mathrm{and}~ 2.4$ for the G.17TC2P0.5, G1.7TC2P1, and G1.7TC2P1.5 disks, respectively. In all cases $p$ was measured in the regions where the Toomre $Q$ and the entropy parameter $K$ were determined to be relatively constant. These determinations were made by eye and cover the ranges of 12 to 48 AU, 12 to 49 AU, and 11 to 45 AU for the G.17TC2P0.5, G1.7TC2P1, and G1.7TC2P1.5 disks, respectively. Considering their initial disparity in $\Sigma(\varpi)$, the final surface density profiles are remarkably similar and also remarkably close to the predicted value.

\begin{figure}[p]
\centering
\unitlength1in
\begin{minipage}[t]{6.3in}
\centering
\includegraphics[scale=.51]{figures/chap3/sigmafinalcompare.eps}
\end{minipage}\\
\begin{minipage}[t]{6.3in}
\centering
\includegraphics[scale=.51]{figures/chap3/qfinalcompare.eps}
\end{minipage}
\begin{minipage}[t]{6.3in}
\centering
\includegraphics[scale=.51]{figures/chap3/kfinalcompare.eps}
\end{minipage}
\caption[Surface density, Toomre $Q$, and entropy for the end of simulations with varying initial surface density profiles]{Plots show the final surface density profile (top), Toomre $Q$ distribution (middle), and entropy parameter $K$ (bottom) versus radius for each of the three disks. The G1.7TC2P0.5 disk is shown in black, G1.7TC2P1 is shown in red, and G1.7TC2P1.5 is shown in blue. The surface density is plotted logarithmically, and initial surface density profiles are presented in dashed lines for comparison. The final entropy profiles are given in code units. The time of measurement is given in ORPs.}
\label{fig:IC:constq}
\end{figure}

To determine the strength of GI activity in the asymptotic phase, I measured $A_m$ for $m=1-64$ from 13 to 19 ORPs for each of the disks. The average $\left< A_m \right>$ values over these time intervals are plotted in figure \ref{fig:IC:Am}. The error bars in this plot represent the RMS fluctuations about the mean for the time interval from 13 to 19 ORPs. Another measure of total non-axisymmetry is the $\left<A_\Sigma\right>$ value (see \S\ref{sec:IC:am}), these values are 2.8, 2.3, and 1.8 for the G1.7TC2P0.5, G1.7TC2P1, and G1.7TC2P1.5 disks, respectively. Although the shape of the $\left< A_m \right>$ spectrum remains the same for the three cases, the non-axisymmetric power decreases as the initial surface density steepens.

\subsection{Conclusions} 

Three main conclusions come from the comparison of the simulations with different initial surface density profiles. First, varying the initial surface density profile makes almost no qualitative difference in the onset and evolution of GI activity. The disks all go through the same evolutionary phases at nearly the same times, and the outcome of the GI activity is very similar in terms of final surface density profile and asymptotic $\left< A_m \right>$. There are, however, measurable differences. As the initial surface density profile steepens, the onset and peak of the burst, as measured by the peak $A_m$ amplitudes, is delayed. Additionally, the mass transport rates during the burst decrease, and the $\left< A_m \right>$ amplitudes during the asymptotic phase decrease slightly as well. 

\begin{figure}[p]
\centering
\includegraphics[scale=0.9]{figures/chap3/amspeccompare.fixed.eps}
\caption[$\left<A_m\right>$ values for the comparison of initial surface density profiles]{Time-averaged values of $A_m$ for each of the three simulations with varied initial surface density profiles. G1.7TC2P0.5 is shown in black, G1.7TC2P1 is shown in red, and G1.7TC2P1.5 is shown in blue. The $A_m$ values are averaged over the interval $13 -19$ ORPs. The ``error bars'' on each $m$-value represent the RMS fluctuations about the time-averaged mean. The plot depicts $m$-values from 1 to 64 on a logarithmic scale.}
\label{fig:IC:Am}
\end{figure}

\section{Variation of Azimuthal Resolution}\label{sec:IC:resolution}

To understand the relationship between azimuthal resolution and the outcome of GIs, I performed four disk simulations, which are all variations of the G1.7TC2P1 simulation. For this study, I concentrate on the effect of resolution on gravitational torques and the effective $\alpha$ in the asymptotic phase of the simulations. Since the computational burden of running all four simulations from the axisymmetric phase is rather high, the $l_{max}$ = 128 was run through the axisymmetric, burst and transition phases. The other simulations all begin by interpolating the $l_{max}$ = 128 simulation to higher or lower azimuthal resolution at 9.6 ORPs, near the end of the transition phase. Although this strategy saves a large amount of computational time, it limits the scope of the analysis to the asymptotic phase. The four resolutions used are $l_{max}$ = 64, 128, 256, and 512. After interpolation, the GIs in the disk have additional or fewer degrees of freedom, and so it takes some time for the disk to transition to the new resolution. I evolved the disk from 9.6 to 12 ORPs before making any comparisons so that the disk can relax to the asymptotic state. All four disks use a volumetric cooling rate as in equation \eqref{eq:constcool} with $t_\mathrm{cool} = 2$. Figure \ref{fig:IC:rescomp} compares the four different resolutions at 18 ORPs near the end of the simulations. One can clearly see that the lower resolution simulations had much more clearly defined spiral structures with greater density contrast at larger radii. The density structures of the higher resolution simulations were much more washed out and are not dominated by low-order structures. This resulted larger non-axisymmetric amplitudes for low-order modes and torques for the simulations with smaller azimuthal resolutions (see figures \ref{fig:IC:Amres} and \ref{fig:IC:torqueres}).

\begin{figure}[p]
\unitlength1in
\begin{minipage}{0.5\linewidth}
\centering
\includegraphics[width=3in]{figures/chap3/LMAX64_18ORP.eps}
\end{minipage}
\hspace{0.25in}
\begin{minipage}{0.5\linewidth}
\centering
\includegraphics[width=3in]{figures/chap3/LMAX128_18ORP.eps}
\end{minipage}
\vspace{0.5in}


\begin{minipage}{0.5\linewidth}
\centering
\includegraphics[width=3in]{figures/chap3/LMAX256_18ORP.eps}
\end{minipage}
\hspace{0.25in}
\begin{minipage}{0.5\linewidth}
\centering
\includegraphics[width=3in]{figures/chap3/LMAX512_18ORP.eps}
\end{minipage}
\caption[Midplane and meridional densities for varied azimuthal resolutions]{The midplane and meridional density contours are shown for each of the four azimuthal resolutions considered. From the top left going clockwise these are $l_{max} = 64$, $128$, $512$, and $256$. Densities are plotted in code units on a logarithmic scale. This snapshot is taken near the end of the simulations at $\approx 18$ ORPs.}
\label{fig:IC:rescomp}
\end{figure}

Azimuthal resolution is particularly important because it allows GI power to spread to higher-order modes (i.e., modes with more arms) that behave more like local turbulence. I test this by examining the amplitude in various modes to see how they are affected by the choice of grid. Figure \ref{fig:IC:Amres} shows the $\left< A_m \right>$ values averaged from 12 to 19 ORPs for the different resolutions. For each simulation, I compute $\left< A_m \right>$ values up to $m_{max} = l_{max}/2$. This represents the smallest wavelength for which I can reliably measure the GI amplitudes \citep{shannon1984}. Clearly, as $l_{max}$ is increased the amplitude increases in modes with a larger $m$-value. Additionally, the amplitudes of low-order $m$-modes is decreased. This is most likely due to the fact that, as $l_{max}$ is increased the degrees of freedom available for the non-axisymmetric structure increase as well, so power is naturally spread from low-order to high-order $m$-values. Table \ref{tbl:IC:amres} lists the time-averaged $\left< A_m \right>$ summed over $m = 2$ to $l_{max}/2$ for each of the resolutions. This total non-axisymmetric amplitude $\left<A_\Sigma\right>$ varies by at most 7\% for the different values of $l_{max}$, with $l_{max} = 256$ being the most significant outlier. The $m = 1$ mode is not included in the integrated $\left<A_\Sigma\right>$ value because the central star is artificially fixed to the grid center in these simulations (see chapter \ref{chap:indirect}). 

I identify $m=2-7$ as the low-order modes that are considered to be global modes, with $m=1$ again excluded due to the fixed central star. This determination was made by considering the radial range over which the mode is most effective in transporting angular momentum, i.e. from the inner Linblad resonance to the outer Linblad resonance, and comparing it to the disk scale height. These quantities are roughly equal for $m=7$. For lower order modes, the radial range exceeds the scale height. Table \ref{tbl:IC:amres} shows that the amount of power present in these low-order modes represents a much smaller fraction of the total power as the azimuthal resolution increases. This fraction systematically decreases as $l_{max}$ increases, with the difference between the lowest resolution and the highest resolution being a factor of 1.6. As $l_{max}$ increases, the percentage change from one resolution to another changes as well. For example, when $l_{max}$ is increased from 64 to 128 $\left<A_{2-7}\right>/\left<A_\Sigma\right>$ decreases by 18\%. However, when $l_{max}$ is increased from 256 to 512 $\left<A_{2-7}\right>/\left<A_\Sigma\right>$ decreases by only 9\%. Ideally, if the simulations were totally converged, the fraction of non-axisymmetric amplitude would not change with an increase in $l_{max}$, however the $l_{max} = 512$ simulation is clearly very close to convergence.

\begin{table}
\centering
\renewcommand{\arraystretch}{1.25}
\begin{tabular*}{0.75\textwidth}{@{\extracolsep{\fill}}ccccc}
\hline
$l_{max}$&$Q_{avg}$&$\left<A_\Sigma\right>$&$\left<A_{2-7}\right>/\left<A_\Sigma\right>$&$\alpha_{avg}$\\
\hline\hline  
64&1.26&1.96&0.76&0.069\\ 
128&1.24&1.91&0.62&0.044\\ 
256&1.39&1.83&0.54&0.026\\ 
512&1.48&1.95&0.49&0.020\\
\hline
\end{tabular*}
\vspace{0.1in}
\caption[Time averaged $A_m$, Toomre $Q$ and $\alpha$ for several azimuthal resolutions.]{Temporally and spatially averaged quantities for several values of $l_{max}$. All values are averaged over the time interval from $12$ to $19$  ORPs. The Toomre $Q$ quantities and the $\alpha$ quantities are averaged in radius from $10$ to $40$ AU. The $\left<A_m\right>$ quantities are averaged from $2$ AU to the outer edge of the grid.}
\label{tbl:IC:amres}
\end{table} 
\renewcommand{\arraystretch}{1}

The total amount of non-axisymmetric power, represented by $\left<A_\Sigma\right>$, is related to the strength of GI activity and stays roughly constant with changing azimuthal resolution. This is due to the fact that the amplitude of non-axisymmetric structures is strongly dependent on the local cooling time \citep{cossins2009}, which is the same for all of the simulations. \citet{cossins2009} find that $\left<A_\Sigma\right>^2 \approx 1/t_{cool}$. This results in the $Q$ values being similar for all of the simulations because the strength of the instability (parameterized by $Q$) is directly related to the amplitude of the non-axisymmetric structure. The Toomre $Q$ values presented in table \ref{tbl:IC:amres} for each of the resolutions are averaged over the time interval 12 to 19 ORPs, and from 10 to 40 AU, the region of the disk that is gravitationally unstable. Like $\left<A_\Sigma\right>$ the time averaged Toomre $Q$ values vary by a relatively small amount as the resolution increases, with the maximum difference being 15\% between $l_{max} = 128$ and 512. Some of the difference may be due to the non-linear and chaotic behavior of the GIs. Additionally, differences in the time sampling may result in minor differences in the time-averaged values. Since the Courant time $\sim \Delta x/v_x$, as the resolution increases, the time step decreases, resulting in a larger number of time samples for the higher-resolution studies, and greater accuracy in the temporal integration. 

Naturally, one would expect as the non-axisymmetric amplitude is transferred from lower-order modes to higher-order modes that the gravitational torques in the disk should decrease. This is due to the fact that low-order modes have a longer wavelength and, if coherent in radius, have a longer lever arm to produce torques. On the other hand, high-order modes have relatively short wavelengths and so produce more localized effects and smaller net torques. These modes tend to produce pockets of local GI activity, whose torques can cancel each other out over the whole disk. Figure \ref{fig:IC:torqueres} compares the torque profiles for the $l_{max} = 128$ and 512 simulations. Contributions to the torque from the $m =1$ pattern, the sum of $m = 1$ and 2, the sum of $m =1$, 2 and 3, and the sum of $m=1$, 2, 3 and 4 are inlcuded in addition to the total torque. As the azimuthal resolution increases from $l_{max} = 64$ to 512, the total torque decreases monotonically. The comparison between $l_{max} = 128 $ and 512 is shown as an example of the general trend. Although the general shape of the torque profile is consistent between the two resolutions, there are some minor variations, primarily due to differences in the torque from $m \geq 4$. 

\begin{figure}[p]
\centering
\subfigure{\includegraphics[scale=0.7]{figures/chap3/amspec_res_zoom.eps}}\\
\subfigure{\includegraphics[scale=0.7]{figures/chap3/amspec_res.eps}}
\caption[$\left<A_m\right>$ values for the comparison of azimuthal resolution]{Time averaged values of $A_m$ for each of the three simulations with varied azimuthal resolution. The $l_{max} =64$ simulation is shown in black, $l_{max} =128$ is shown in blue, $l_{max} = 256$ is shown in green, and $l_{max} =512$ is shown in red. The $A_m$ values are averaged of the interval from $12-19$ ORPs. The ``error bars'' in the lower plot on each $m$ value represent the RMS fluctuations about the time averaged mean. The top plot depicts a zoomed in region for $m = 2-10$, it is shown without ``error bars'' for clarity. The plots depict $m$-values from $2$ to $l_{max}/2$ on a logarithmic scale.}
\label{fig:IC:Amres}
\end{figure}

Figure \ref{fig:IC:alphares} shows the effective $\alpha$ for the four different grid resolutions, time-averaged from 12 to 19 ORPs.  For comparison, the predicted values from \citeapos{gammie2001} equation \eqref{eq:IC:gammiealpha} are shown for the strongly self-gravitating limit (lower curve) and the non-self-gravitating limit (upper curve).   Note that predictions from \citet{gammie2001} are based upon a thin disk approximation, where the local balance of heating and cooling dominates the energetics.

\begin{figure}[p]
\centering
\includegraphics[scale=0.9]{figures/chap3/torque_res.eps}
\caption[Comparison of $l_{max} = 128$ and 512 run asymptotic torque profiles.]{Torque profiles averaged from $12$ to $19$ ORPs for the $l_{max} =512$ (red) and $l_{max} =128$ (black) simulations. Total torques are shown as well as contributions from sums of various $m$-values.}
\label{fig:IC:torqueres}
\end{figure} 

In an $\alpha$-disk formalism, the mechanism giving rise to the transport of angular momentum also acts to dissipate energy. Based on simulations using a local cooling prescription, i.e. $t_{cool} \Omega = \mathrm{const.}$, \citet{lodato2005} concluded that a disk in a state of gravitoturbulence would have locally balanced heating and cooling rates. If the heating were due to local dissipation, then $\alpha$ and $t_{cool}$ are related by
\begin{equation}\label{eq:IC:alphatcool}
\alpha = \left | \frac{\dif \ln \Omega}{\dif \ln R}\right|^{-2} \frac{1}{\gamma (\gamma -1) t_{cool} \Omega}.
\end{equation}
\citet{rice2005} concluded that for a given $t_{cool} \Omega$ there exists some maximum stress  $\alpha_{max}$, which if exceeded would cause a disk to fragment. The horizontal dot-dashed line in figure \ref{fig:IC:alphares} indicates the $\alpha_{max} \approx 0.06$ found by \citet{rice2005}.  \citet{clarke2007} have revised this limit to $\alpha_{max} = 0.12$ for cases where $t_{cool}$ varies with time, but, because the $t_{cool}$ used in these simulations is constant with time, I display the $\alpha_{max}$ presented by \citeauthor{rice2005}. It is interesting to note that at higher resolution the disks do not reach this critical $\alpha$ and do not fragment. However, the lower resolution disks that do exceed this threshold do not fragment either. This could be due to suppression of fragmentation at lower resolution because there is insufficient resolution to follow the fragments as they form.

\begin{figure}[p]
\centering
\includegraphics[scale=0.9]{figures/chap3/alpha_res.eps}
\caption[Comparison of $\alpha$ profiles for varying azimuthal resolution]{Effective Shakura-Sunyaev $\alpha$-values computed for the $l_{max} = 64$ (black), $128$ (blue), $256$ (green), and $512$ (red) simulations averaged over the asymptotic phase from $12$ to $19$ ORPs. Shown for comparison are curves predicted by \citeauthor{gammie2001} with $t_{cool} = 2$ ORPs. The upper curve assumes a self-gravitating disk; the lower curve assumes gravity is due to the star alone. The dot-dash line indicates the critical $\alpha$ found by \citet{rice2005} for fragmentation.}
\label{fig:IC:alphares}
\end{figure}

As the azimuthal resolution increases, the effective $\alpha$ in the GI active region decreases. This is due to the decrease in the total gravitational torque, as illustrated in figure \ref{fig:IC:torqueres}. This decrease in gravitational torque is due, in turn, to the shift of non-axisymmetric amplitude from low-order global modes, which produce large torques, to high-order local modes, which tend to cancel each other out more when integrated over the whole disk. In fact, the decrease in the ratio of low-order amplitude to the total amplitude $\left<A_{2-7}\right>/\left<A_\Sigma\right>$ closely tracks the decrease in the time-averaged $\alpha$ averaged from 10 to 40 AU, as seen in table \ref{tbl:IC:amres}. Although the average $\alpha$ values have not converged to a single value for my highest two resolutions, the percentage change decreases as $l_{max}$ increases. In addition, figure \ref{fig:IC:alphares} shows that the $l_{max} = 256$ and 512 simulations have qualitatively similar time averaged $\alpha$ profiles over a large range of disk radii. These profiles also fall midway between the predicted \citeauthor{gammie2001} curves. Indeed the $\alpha_{avg}$ value of 0.020 for the $l_{max} = 512$ simulation is squarely between the $\alpha_{avg}$ values for the \citeauthor{gammie2001} curves 0.014 (strongly self-gravitating) and 0.027 (non-self-gravitating). 

%%%%%%% FIX THIS!!!!!! %%%%%%%%
Clearly, azimuthal resolution plays an important role in accurately quantifying the key mass transport mechanisms in global simulations of gravitationally unstable disks. Although I have not conclusively shown numerical convergence at a resolution of $l_{max} = 512$, there are strong indications that this resolution is adequate to appropriately quantify the magnitude of gravitational torques and an effective gravitational $\alpha$ resulting from GI activity. The percent change in $\alpha_{avg}$ between $l_{max} = 256$ and 512 is the lowest of any resolution increase, and although it is a $\sim 20\%$ difference the qualitative difference between the $l_{max} = 256$ and 512 $\alpha_{avg}$ curves is rather small. Of course, to conclusively determine convergence would require another simulation with a $l_{max}$ of 1024, but the computational cost is prohibitive for this case at the present time.

In addition, the $\alpha$ curve for $l_{max} = 512$ falls squarely between the \citeauthor{gammie2001} predicted curves. Gammie's prediction is based on the idea of ``gravitoturbulence" where the only assumption is a local balance of cooling, in this case from the $t_{cool}$ prescription, and heating from GI activity. If the heating from GI activity is localized then the local balance of heating and cooling assumption should be valid for these disks. However, it should be noted that Gammie's predictions are not correct on all counts (see below), and are only valid when the quantities considered are averaged over long time periods and considered over a large radial range. If one were to use the parametrization predicted by Gammie's formula in simulations, much of the detailed structure would be missed. Such local, rapidly varying structure can play an important role in many disk processes (see, for example, planet migration in chapter \ref{chap:planet}). In addition, to correctly predict the $\alpha$ in a disk using Gammie's formula, one must know the disk cooling time, which, in the case of a disk with realistic radiative cooling, is not known {\it a priori}.

Although the effective $\alpha$ curves do correspond roughly to the \citeauthor{gammie2001} prediction, the results from these simulations differ from \citeauthor{gammie2001}'s in one significant way. The local shearing box simulations performed by \citet{gammie2001} show that the total effective $\alpha$ is composed of a gravitational stress component and a Reynolds stress component that are nearly equal, see \citet[figure 3]{gammie2001}. On the other hand, \citet{lodato2004} reported similar small Reynolds stresses for several of their simulations (see their figure 5). All of the analyses presented here only consider the gravitational stress. When I attempt to measure the Reynolds stress, I find it is quite small compared to the gravitational stress. As noted previously, the Reynolds stress is very hard to determine in global simulations. However, the prediction made by \citeauthor{gammie2001} relies only on the idea that heating from GI activity balances cooling. So the exact mix of stresses contributing to the total amount of heating could be different depending on the type of simulation, the physics considered in the simulation, and the simulation parameters. 

The requirement of $l_{max} =512$ as a minimum azimuthal resolution for accurate computation of GI effects does not necessarily apply to all possible disk analyses. This study focused primarily on the accurate determination of non-axisymmetric structure, gravitational torques, and effective gravitational $\alpha$. Another aspect where azimuthal resolution is known to be important is disk fragmentation \citep{pickett2003}. However, comparative studies, such as the one presented in \S\ref{sec:IC:surfden}, still produce valid results at lower azimuthal resolutions.

\section{Variation of $\gamma$ in the Equation of State}\label{sec:IC:gamma}

All the simulations discussed in this chapter use an ideal gas equation of state as outlined in chapter \ref{chap:numeth}. The pressure is determined by the internal energy $\epsilon$ and the adiabatic index $\gamma$ as in \eqref{eq:energyeq}. As the adiabatic index decreases, so does the pressure for a given $\epsilon$, and this is typically referred to as a ``softer'' equation of state because the fluid is more compressible. Authors who use an ideal gas equation of state with fixed $\gamma$ have employed both $\gamma = 7/5$, because it mimics molecular hydrogen with excited rotation states \citep{boss1998b}, and $\gamma = 5/3$, because it represents a monatomic ideal gas \citep{pickett1998}. In fact, molecular hydrogen \cf{H2} can behave like a $\gamma = 7/5$ or $\gamma =5/3$ ideal gas depending on the composition and temperature. If one disregards dissociation and ionization, \cf{H2} typically behaves like a $\gamma = 5/3$ ideal gas for $T < 100K$ and like a $\gamma = 7/5$ ideal gas for $T > 300K$ \citep{decampli1978}. The precise temperature range over which this transition takes place depends on the ratio of ortho-hydrogen (parallel proton spins) to para-hydrogen (antiparallel proton spins) \citep{boley2007a}.

Naturally, one would assume that a lower $\gamma$, and thus a more compressible fluid, would lead to a protoplanetary disk that is more susceptible to fragmentation. Indeed, in a series of Smoothed Particle Hydrodynamic (SPH) simulations, \citet{rice2005} found that disks with $\gamma = 7/5$ were susceptible to fragmentation at larger $t_{cool} \Omega$ than disks with $\gamma =5/3$; specifically, the fragmentation limit was $t_{cool} \Omega = 12-13$ for $\gamma = 7/5$ and $t_{cool}\Omega = 6-7$ for $\gamma = 5/3$. To confirm the findings of \citeauthor{rice2005}, and to test the applicability for fixed grid simulations with global cooling times, I performed four simulations in addition to those already discussed. These are G1.6TC0.6P1, G1.4TC3P1, G1.4TC2P1, and G1.4TC1P1. To produce initial equilibrium disk models for the $\gamma = 7/5$ simulations, I modified the initial axisymmetric equilibrium models used as initial models for the $\gamma = 5/3$ simulations. To maintain the pressure balance, and thereby the hydrostatic equilibrium, of the initial disk I changed the internal energy density, $\epsilon$, of each cell according to equation \eqref{eq:idealpres} so that, 
\begin{equation}
\epsilon_{7/5} = 5/3 \epsilon_{5/3}.
\end{equation}
Here $\epsilon_{5/3}$ and $\epsilon_{7/5}$ are the internal energy densities of the disks with $\gamma = 5/3$ and $7/5$, respectively. Although this change of internal energy density maintains pressure equilibrium, it introduces an entropy gradient that is dynamically unstable. However, the fluid re-equilibrates rapidly (in $< 1$ ORP) and stability is quickly restored.

\begin{figure}[p]
\unitlength1in
\begin{minipage}{0.5\linewidth}
\centering
\includegraphics[width=2.4in]{figures/chap3/G1.6TC.6_ring.eps}
\end{minipage}
\hspace{0.25in}
\begin{minipage}{0.5\linewidth}
\centering
\includegraphics[width=2.4in]{figures/chap3/G1.4TC1_ring.eps}
\end{minipage}
\vspace{0.2in}

\begin{minipage}{0.5\linewidth}
\centering
\includegraphics[width=2.4in]{figures/chap3/G1.6TC.6_frag.eps}
\end{minipage}
\hspace{0.25in}
\begin{minipage}{0.5\linewidth}
\centering
\includegraphics[width=2.4in]{figures/chap3/G1.4TC1_frag.eps}
\end{minipage}
\vspace{0.2in}

\centering
\begin{minipage}{0.5\linewidth}
\centering
\includegraphics[width=2.4in]{figures/chap3/G1.4TC2_frag.eps}
\end{minipage}
\caption[Midplane and meridional densities of fragmenting disks]{Midplane and meridional densities are plotted in code units on a logarithmic scale for the G1.7TC0.6P1 (left) and G1.4TC1P1 (right) simulations. The top panels show both simulations just before fragmentation, a dense ring has formed as a precursor to fragmentation. The second row of panels show the simulations just after the disk has fragmented, measurement of the fragments positions determines the $t_{cool}\Omega$ at which each disk fragments. The bottom panel shows an example of a tenuous fragment that forms in the G1.4TC2P1 simulation. This fragment is subsequently destroyed by Keplerian shear.}
\label{fig:IC:gammaden}
\end{figure}

It has been shown that the primary factor that governs the strength of GIs in a disk is the cooling rate \citep{gammie2001,mejia2005}. For $\gamma = 5/3$, \citet{gammie2001} found that a disk should fragment if $t_{cool}\Omega \lesssim 3$ and \citet{mejia2005} found that a disk will fragment for a global cooling rate $t_{cool} \lesssim 0.5$ ORP, provided the simulation is carried out at high enough resolution. The procedure I employed was to perform a simulation using a cooling rate that would not cause the disk to fragment, and then run subsequent simulations, decreasing the cooling rate until the disk fragmented. For a $\gamma = 5/3$ gas, I found that a disk will fragment with a global cooling rate of $t_{cool} = 0.6$ ORPs. With a $\gamma = 7/5$, gas I found that a global cooling rate of $t_{cool} = 1$ ORPs will cause strong fragmentation. When $t_{cool} = 2$ ORPs and the adiabatic index is $7/5$, I found that the disk is on the verge of fragmentation. Fragments form, but are quickly destroyed by Keplerian shear, see bottom panel of figure \ref{fig:IC:gammaden} for an example. All four simulations were identical to the G1.7TC2P1 simulation except for the adiabatic index, cooling rate, and in the case of fragmenting disks, the azimuthal resolution. Both fragmenting disks, G1.7TC0.6P1 and G1.4TC2P1, and the borderline case, G1.4TC2P1, were simulated with $l_{max} = 512$.

Figure \ref{fig:IC:gammaden} shows midplane and meridional densities for the two strongly fragmenting simulations, G1.7TC0.6P1 and G1.4TC1P1. Both disks produce many tightly bound clumps, but the disk with a softer equation of state does so with a longer cooling time. Although the determination of the fragmentation threshold is not precise, it does set a maximum global $t_{cool}(crit)$ that will cause fragmentation in the disk. Additionally, by examining the location in the disk where the fragmentation occurs, I can convert the global $t_{cool}(crit)$ to a local $t_{cool}\Omega  (crit)$. In both of the fragmenting cases, a dense ring forms at the radius where the fragmentation occurs, shortly before the disk breaks apart into fragments. The dense ring can be seen forming in the top panels of figure \ref{fig:IC:gammaden}, and the middle panels show the time when clumps first appear. By measuring the average $\Omega$ at the radius where clumps appear, I can determine the value of $t_{cool}\Omega (crit)$ for which the fragmentation occurs. 

\begin{figure}[p]
\centering
\includegraphics[scale=0.9]{figures/chap3/tcomega.eps}
\caption[The $t_{cool} \Omega$ curves for fragmenting simulations]{The $t_{cool}\Omega$ curves are plotted versus radius for the G1.4TC1P1 (red) and G1.7TC0.6P1 (black) simulations. For comparison the $t_{cool}(crit)\Omega$ lines from \citet{rice2005} are plotted for $\gamma = 7/5$ and $5/3$. The vertical hash mark represents the radius at which fragments first form in each of the simulations.}
\label{fig:IC:tcomega}
\end{figure}

Figure \ref{fig:IC:tcomega} shows $t_{cool}\Omega$ plotted as a function of radius for the G1.7TC0.6P1 and G1.4TC1P1 simulations. The simulation with $\gamma = 7/5$ is shown in red, while the $\gamma = 5/3$ simulation is shown in black. The dashed lines indicate the fragmentation criteria as a function of $t_{cool}\Omega$ presented by \citet{rice2005} as $t_{cool}\Omega (crit) = 12-13$ for $\gamma = 7/5$ and $t_{cool}\Omega (crit) = 6-7$ for $\gamma = 5/3$. The minimum radius where fragmentation occurs in each of the simulations is shown with a hash mark. I find fragmentation criteria of about $t_{cool}\Omega (crit) = 5.5$ and 11 for $\gamma = 5/3$ and $7/5$, respectively. These numbers are in rough agreement with those found by \citeauthor{rice2005}. 

\section{Variation of the Initial Perturbation}\label{sec:IC:randpert}

\begin{figure}[p]
\centering
\unitlength1in
\begin{minipage}[t]{6.3in}
\centering
\includegraphics[scale=.5]{figures/chap3/amgrowthP1_25AU.eps}
\end{minipage}\\
\begin{minipage}[t]{6.3in}
\centering
\includegraphics[scale=.5]{figures/chap3/amgrowthP1RAND_25AU.eps}
\end{minipage}
\begin{minipage}[t]{6.3in}
\centering
\includegraphics[scale=.5]{figures/chap3/amgrowthP1RAND5_25AU.eps}
\end{minipage}
\caption[Fourier components as a function of time for different initial random perturbation amplitudes]{Each panel shows the Fourier components, $A_m$, as a function of time for $m = 1 - 6$. The $\xi = 0.0001$, $0.01$, and $0.05$ simulations are shown from top to bottom. The amplitudes are shown from the initial models through the non-linear saturation in the burst phase. Contribution to the $A_m$ components is only calculated for $\varpi > 25$ AU due to contamination from a spurious $m = 1$ signal.}
\label{fig:IC:amtpert}
\end{figure}

In order to seed the non-axisymmetric structure that will eventually grow to a non-linear amplitude, a small initial random perturbation is introduced into the axisymmetric equilibrium models used to begin a simulation. Typically, this perturbation is rather small with the amplitude being $\xi = 0.0001$, i.e. $\Delta \rho /\rho = \xi R$, where $R$ is a random number between $-1$ and 1. The perturbation $\Delta \rho /\rho$ is calculated for each cell and $R$ is uncorrelated from cell to cell. This value of $\xi$ is used for all simulations discussed in this work other than those in this section. In order to investigate the effect this random perturbation has on the onset of GIs, I performed two additional simulations by giving the G1.7TC2P1 initial axisymmetric equilibrium disk random perturbations with amplitude $\xi = 0.01$ and $\xi = 0.05$. These simulations were run through the burst phase using $l_{max} =128$. 

Figure \ref{fig:IC:amtpert} shows the evolution of the $A_m$ values over time for $m = 1 -6$ for each of the three simulations. As in figure \ref{fig:IC:amt}, only material at $\varpi > 25$ AU contributes to the measurement of $A_m$. As the value of $\xi$ is increased, the initial $A_m$ amplitudes increase as well. However, this change also results in a more subtle effect. For larger $\xi$, the growth in amplitude begins almost immediately, but at a somewhat slower rate. Eventually, the growth steepens, but less and less so as $\xi$ gets larger. The net effect is that the $A_m$ values saturate somewhat sooner with the {\it smallest} value of $\xi$. 

\begin{figure}[p]
\centering
\unitlength1in
\begin{minipage}[t]{6.3in}
\centering
\includegraphics[scale=.375]{figures/chap3/P1_den.eps}
\end{minipage}\\
\begin{minipage}[t]{6.3in}
\centering
\includegraphics[scale=.375]{figures/chap3/RAND_den.eps}
\end{minipage}
\begin{minipage}[t]{6.3in}
\centering
\includegraphics[scale=.375]{figures/chap3/RAND5_den.eps}
\end{minipage}
\caption[Midplane and meridional densities for varying initial random perturbation amplitudes]{Midplane and meridional densities are shown in code units on a logarithmic scale for the $\xi = 0.0001$ (top), $0.01$ (middle), and $0.05$ (bottom) simulations. Densities are shown at $4.25$ ORPs in the middle of the $A_m$ growth, when non-axisymmetric structure is just beginning to be visible. Note the lack of non-axisymmetric structure in the $\xi = 0.0001$ simulation.}
\label{fig:IC:denpert}
\end{figure}

This phenomenon can be explained in a fairly straightforward manner. The larger the value of $\xi$, the larger the initial $A_m$ amplitudes. This means that significant coherent non-axisymmetric structures appear very early in the simulations with large $\xi$ values. Figure \ref{fig:IC:denpert} shows the midplane and meridional densities of each of the three simulations at 4.25 ORPs. In the simulations with $\xi = 0.01$ and 0.05 one can see non-axisymmetric structure, while the simulation with $\xi = 0.0001$ remains nearly axisymmetric. This non-axisymmetric structure acts to heat the disk and thereby prevents $Q$ from dropping rapidly. The axisymmetric ring that grows as a precursor to the burst of GI activity in the $\xi = 0.0001$ simulations is disrupted by the non-axisymmetric structures present earlier in the $\xi = 0.01$ and 0.05 simulations, and so the amplitude continues to grow at a slower rate. When the axisymmetric ring forms in the $\xi = 0.0001$ simulation, $Q$ becomes very small, in some cases less than 1, and so a strong burst of GI activity is initiated at the radius where the ring forms. Figure \ref{fig:IC:pertq} shows the surface density profiles and Toomre $Q$ profiles for the three disks near the same time as the density snapshots in \ref{fig:IC:denpert}. The surface density is presented in linear scale to highlight the differences in the surface density where the dense ring forms. This ring has an inner radius near 25 AU and is centered around 30 AU. It is most prominent in the simulation with $\xi = 0.0001$ and decreases in mass as $\xi$ increases. As figure \ref{fig:IC:pertq} shows, this overdensity can be correlated to a sharp decrease in the Toomre $Q$. The $Q$ profile abruptly changes slope near 24 AU for the $\xi = 0.0001$ simulation corresponding to the inner radius of the ring. The change in slope is less pronounced for the $\xi = 0.01$ simulation and is not noticeable for the $\xi = 0.05$ simulation. As the additional mass in the ring forces $Q$ to smaller values, the GI activity increases. This, in turn, appears as rapid growth in the $A_m$ amplitudes.

\begin{figure}[p]
\centering
\subfigure{\includegraphics[scale=0.7]{figures/chap3/randsurfdencomp.eps}}\\
\subfigure{\includegraphics[scale=0.7]{figures/chap3/randqcomp.eps}}
\caption[Surface density and Toomre $Q$ for the simulations with varying initial random perturbations at 4 ORPs]{Plots show the final surface density profile (top) and Toomre $Q$ distribution (bottom) versus radius for each of the three disks. The G1.7TC2P1 disk is shown in black, G1.7TC2P1RAND1 is shown in red, and G1.7TC2P1RAND5 is shown in blue. The time of measurement is given in ORPs.}
\label{fig:IC:pertq}
\end{figure}

Another interesting difference between the simulations with small and large $\xi$ values is the fastest growing mode. For both $\xi = 0.01$ and 0.05, $m=3$ is the dominate mode while the modes are growing, as shown in figure \ref{fig:IC:amtpert}. The $m = 2$ mode is also more prominent than in the small $\xi$ simulation. For $\xi = 0.0001$, $m = 4$ and 5 dominate. This may be due to the fact that, when an axisymmetric ring is allowed to form, $Q$ is at its minimum over a very small radial extent, which may bias the disk towards the growth of modes with shorter wavelengths. However, when non-axisymmetric structures are present over a large radial extent and $Q$ is roughly the same over this range, modes of any wavelength can grow.

Although a variation of perturbation amplitudes did cause significant differences in the burst phase, the early asymptotic phase is largely unaffected. In figure \ref{fig:IC:pertq15} one can see the surface density profile and Toomre $Q$ profile of each of the disks at $\approx 15$ ORPs. There is some minor variation in the inner disk at radii $< 10$ AU with different surface density enhancements, but the inner disk, specifically the radius at which the disk transitions from being GI active to inactive, is very chaotic. For radii $> 15$ AU there is very little difference between the disks in either surface density or $Q$ value.

\begin{figure}[p]
\centering
\subfigure{\includegraphics[scale=0.7]{figures/chap3/sigmacompare.15ORP.eps}}\\
\subfigure{\includegraphics[scale=0.7]{figures/chap3/randqcomp.15ORP.eps}}
\caption[Surface density and Toomre $Q$ for the simulations with varying initial random perturbations at 15 ORPs]{Plots show the final surface density profile (top) and Toomre $Q$ distribution (bottom) versus radius for each of the three disks. The G1.7TC2P1 disk is shown in black, G1.7TC2P1RAND1 is shown in red, and G1.7TC2P1RAND5 is shown in blue. The time of measurement is given in ORPs.}
\label{fig:IC:pertq15}
\end{figure}

\section{Conclusions}\label{sec:IC:conclusion}

In this chapter, I have explored how varying both initial conditions and simulation parameters can affect the onset and outcome of gravitational instabilities in protoplanetary disks. In a disk with a constant global cooling time, the onset of GIs occurred at approximately the same radius, so for a steeper initial surface density profile less mass was present in the GI-active region. This means that the burst of GI activity took longer to occur and was less effective at transporting mass inward. Nevertheless, overall the disk evolution was largely unchanged. All disks followed the same pattern of evolutionary phases, and when they reached the asymptotic phase they had all converged to similar configurations, namely, a GI active protoplanetary disk with roughly constant $Q$ and constant specific entropy with a surface density profile of $\Sigma \propto \varpi^{-5/2}$.

When considering the effect of azimuthal resolution on the asymptotic behavior of a GI unstable disk, I found that, without sufficient azimuthal resolution ($l_{max} \leq 256$), gravitational torques and the effective gravitational $\alpha$ were underestimated for disks with a constant global cooling time. I also found that, at higher resolutions, non-axisymmetric amplitude shifted to higher-order modes, and the amplitude of low-order modes decreased. This resulted in a decrease of the gravitational torque and effective gravitational $\alpha$. For $l_{max} =512$, the effective gravitational $\alpha$ was in very good agreement with the prediction of \citet{gammie2001}, which follows from an argument of local balance of GI heating and cooling. However, his detailed simulations found that the torques and heating in the disk were due in equal parts to gravitational stresses and Reynolds stresses. This is in contrast to my measurements where only gravitational stresses were significant, in rough agreement with \citet{lodato2004}. 

I also found agreement with \citet{rice2005} on the $t_{cool}\Omega (crit)$ fragmentation criteria for disks with adiabatic indices of $\gamma = 5/3$ and $7/5$. The key finding was that the trend found by \citeauthor{rice2005} for constant $t_{cool}\Omega$ cooling also applies for a constant global cooling time. Perhaps not surprisingly, the more compressible $\gamma =7/5$ equation of state led to fragmentation at longer cooling times, twice as long as for a disk composed of a $\gamma = 5/3$ gas. 

Finally, my simulations studying the amplitude of the initial random perturbation showed that the extremely small amplitudes used in most simulations allow disks with constant global cooling rates to develop dense axisymmetric rings before a burst of GI activity. For the large variety of environments in which a protoplanetary disk may be found, it is likely that protoplanetary disks have a variety of initial perturbations. A disk with a $\gtrsim$ 1\% amplitude for random density perturbations avoids the artificially low $Q$ values found in the axisymmetric ring. It should be noted that disks with larger perturbations still followed the same basic phases of evolution, but the initial growth of the instability was not as dramatic as in simulations with small perturbations. In addition, the primary spiral modes seen in the burst phase were affected by the amplitude of the initial perturbation.
