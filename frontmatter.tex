%
%Front matter for Scott Michael's thesis
%2003
%includes:
%         title page
%         acceptance page
%         copyright page
%         dedication page
%         Acknowledgements
%         Abstract (with commented commands on
%                   created UMI abstract)
%         Table of Contents
%         List of Figures
%         List of Tables
%

%
%\begin{Title Page}
%

\pagestyle{empty}
\pagenumbering{roman}
\begin{spacing}{1.0}

\hbox{}
\vspace{1.0in}

\begin{center}
\begin{Large}
\begin{spacing}{1.2}
\textbf{PLANET MIGRATION INDUCED BY GRAVITATIONAL INSTABILITIES}
\end{spacing}
\end{Large}

\vspace{2.0in}

\begin{large}
Scott Michael
\end{large}

\vspace{2.25in}

Submitted to the faculty of the University Graduate School \\
in partial fulfillment of the requirements \\
for the degree \\
Doctor of Philosophy \\
in the Department of Astronomy, \\
Indiana University \\
August 2010
\end{center}

\end{spacing}

\clearpage

%
%\end{Title Page}
%

\pagestyle{plain}
\pagenumbering{roman}
\setcounter{page}{2}

%
%\begin{Acceptance Page}
%

\vspace*{1.0in}
\begin{spacing}{1.0}
Accepted by the Graduate Faculty, Indiana University, in partial
fulfillment of the requirements for the degree of Doctor of
Philosophy.
\vspace{1.25in}

\noindent
\hbox{\hspace{2.4in} \underbar{\hbox{\hspace{3.25in}}}} \\
\hbox{\hspace{2.4in}Richard H. Durisen, Ph.D.}

\vspace{0.75in}

\noindent
\hbox{\hspace{2.4in} \underbar{\hbox{\hspace{3.25in}}}} \\
\hbox{\hspace{2.4in} Stuart L. Mufson, Ph.D.}

\vspace{0.75in}

\noindent
\hbox{\hspace{2.4in} \underbar{\hbox{\hspace{3.25in}}}} \\
\hbox{\hspace{2.4in} Haldan N. Cohn, Ph.D.}

\vspace{0.75in}

\noindent
\hbox{\hspace{2.4in} \underbar{\hbox{\hspace{3.25in}}}} \\
\hbox{\hspace{2.4in} Charles J. Horowitz, Ph.D.}

\vspace{1.0in}

\noindent
August 4, 2003
%\today
\end{spacing}

\clearpage

%
%\end{Acceptance Page}
%
%\begin{Copyright Page}
%

\vspace*{3.5in}

\begin{spacing}{1.0}
\begin{center}
Copyright {\copyright} 2011 \\
Scott Michael \\
%ALL RIGHTS RESERVED
\end{center}
\end{spacing}

\clearpage

%
%\end{Copyright Page}
%
%\begin{Dedication Page}
%

\vspace*{2.5in}
\begin{center}
For Rhonda \\
{\it my one true love}
\end{center}
\clearpage

%
%\end{Dedication Page}
%

\begin{spacing}{1.66}

%
%\begin{Acknowledgements}
%
\chapter*{Acknowledgements}

%\begin{center}
%{\large\bf Acknowledgements}
%\vspace{0.05in}




\clearpage

%
%\end{Acknowledgements}
%
%\begin{Abstract}
%

\begin{center}
Scott Michael

\textbf{PLANET MIGRATION INDUCED BY GRAVITATIONAL INSTABILITIES}
%{\large\bf Abstract}
%\vspace{0.05in}

\end{center}

% Extra text for UMI Abstracts
%  \vspace{0.25in}
%  \begin{spacing}{1.0}
%  Constantine P. Deliyannis, Ph.D. \\
%  \end{spacing}
%  \vspace{0.25in}

%\noindent
\begin{spacing}{1.24}
ABSTRACT: The study of gravitational instabilities (GIs) and their effects in protoplanetary disks has been an area of active research for over a decade. Although some studies have indicated that GIs cannot form gas giant planets directly, it is clear that they can have a significant effect on a protoplanetary disk. In this dissertation I present several areas where GIs may play a key role in the evolution of a protoplanetary disk. These studies are carried out using three dimensional numerical simulations. I have carried out and analyzed nearly twenty simulations with varying initial conditions, resolutions, and physical effects. Although all indications from these simulations are that GIs cannot form gas giant planets directly at radii smaller than 40 AU, they have shown that GIs can have a dramatic effect on protoplanetary disk structure and planets embedded in a protoplanetary disk. 

I present several key results including: the effects of a varied initial surface density profile, azimuthal resolution, the amplitude of the initial random perturbation, and the adiabatic index used on the onset, strength and general evolution of GIs in protoplanetary disks. Additionally, I present results on studies of the interaction of the instabilities with the central star when it is allowed to move freely in response to the action of the GIs. Finally, I present several results regarding the interaction of embedded massive planets and GI active disks. I find that the presence of massive planets can have a dramatic effect on the evolution of GIs in an active disk, and the GIs can also dramatically effect them migration of the embedded planet. In fact, the action of the GIs may planets to migrate outward, contrary to the standard theory of the planet migration in laminar disks.
\end{spacing}


\vspace{0.3125in}

\begin{spacing}{1}
\noindent
\hbox{\hspace{2.5in} \underbar{\hbox{\hspace{3.25in}}}} \\

\vspace{0.0125in}

\noindent
\hbox{\hspace{2.5in} \underbar{\hbox{\hspace{3.25in}}}} \\

\vspace{0.0125in}

\noindent
\hbox{\hspace{2.5in} \underbar{\hbox{\hspace{3.25in}}}} \\

\vspace{0.0125in}

\noindent
\hbox{\hspace{2.5in} \underbar{\hbox{\hspace{3.25in}}}} \\
\end{spacing}

\clearpage

%\end{Abstract}

%\begin{ToC, LoF, LoT}

\begin{spacing}{1.24}

\tableofcontents
\listoffigures
\listoftables

\end{spacing}

\clearpage

%\end{ToC, LoF, LoT}

\end{spacing}
