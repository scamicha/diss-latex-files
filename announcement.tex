%-*-LaTeX-*-
\documentclass[letterpaper,11pt]{article}

% Aaron Steinhauer
% Indiana University Astronomy Department
% Email:  aaron@astro.indiana.edu

% Ph.D. Dissertation Announcement

%
%all the LaTeX formatting and definition crap
%for Aaron Steinhauer's thesis
%2003
%Stolen mostly from Adam Rengstorf and Shawn Slavin
%

\usepackage[figuresright]{rotating}
\usepackage{amsmath}
\usepackage{amssymb}
\usepackage{mhchem}
%\usepackage{doublespace}
\usepackage{setspace}
\usepackage{natbib}
%\usepackage{graphics}
%\usepackage{deluxetable}
%\usepackage{mydefs}
\usepackage{afterpage}
\usepackage{float}
%\usepackage{xspace}
\usepackage{verbatim}
\usepackage{graphicx}
\usepackage{grffile}
\usepackage{epstopdf}
\usepackage[font=small,format=plain,labelfont=bf,up,textfont=it,up]{caption}
\usepackage{subfigure}
\usepackage{geometry}   %SM
%\usepackage{xcolor}
%\usepackage{dcolumn}

\setcounter{secnumdepth}{3}
\setcounter{tocdepth}{2}


% Replaced setlengths with geometry call

\geometry{papersize={8.5in,11in},textwidth=5.95in,textheight=8.3in,left=1.525in,top=1.525in,headheight=0.15in,headsep=0.35in,marginparsep=0.125in,marginparwidth=0.75in,verbose}

% \special{papersize=8.5in,11in}
% \setlength{\oddsidemargin}{0.525in}
% \setlength{\evensidemargin}{0.525in}
% \setlength{\topmargin}{0.025in}
% \setlength{\textwidth}{5.95in}
% \setlength{\textheight}{8.3in}
% \setlength{\headheight}{0.15in}
% \setlength{\headsep}{0.35in}
% \setlength{\marginparsep}{0.125in}
% \setlength{\marginparwidth}{0.75in}
% \setlength{\footskip}{0.125in}

\makeatletter
\renewcommand{\@biblabel}[1]{\hspace{-\itemsep}}
%\let\ps@plain=\ps@headings
\setlength\@fpsep{2in}
\makeatother

\makeatletter
\renewcommand*\@makechapterhead[1]{%
  \vspace*{5\p@}%
  {\parindent \z@ \raggedright \normalfont
    \ifnum \c@secnumdepth >\m@ne
        \huge\bfseries \@chapapp\space \thechapter
        \par\nobreak
        \vskip 15\p@
    \fi
    \interlinepenalty\@M
    \Huge \bfseries #1\par\nobreak
    \vskip 35\p@
  }}
\renewcommand*\@makeschapterhead[1]{%
  \vspace*{5\p@}%
  {\parindent \z@ \raggedright
    \normalfont
    \interlinepenalty\@M
    \Huge \bfseries  #1\par\nobreak
    \vskip 35\p@
  }}
\makeatother

%\end{Thanks Shawn}

\def\citeapos#1{\citeauthor{#1}'s (\citeyear{#1})}
\renewcommand{\bibname}{References}
\renewcommand{\floatpagefraction}{0.5}
\renewcommand{\textfloatsep}{0.2in}
%\floatplacement{deluxetable}{tbp}
%\renewcommand{\HII}{\ion{H}{2}\xspace}    %  HII
\bibpunct{(}{)}{;}{a}{}{,}
%\newcommand{\feh}{$[Fe/H]$\xspace}
\newcommand{\dif}{\mathrm{d}}

%\newcolumntype{d}[0]{D{.}{.}{-1}}

%\newcommand\teff{$T_{eff}$\xspace}

%include figure commands
\def\eps@scaling{1.0}% 
\newcommand\epsscale[1]{\gdef\eps@scaling{#1}}% 
% \newcommand\plotone[1]{% 
%  \centering 
%  \leavevmode 
% % \includegraphics[width={\eps@scaling\columnwidth}]{#1}% 
%  \resizebox{\columnwidth}{!}{\includegraphics{#1}}% 
% }% 
% \newcommand\plotport[1]{% 
%  \centering 
%  \leavevmode 
% % \includegraphics[width={\eps@scaling\columnwidth}]{#1}% 
%  \resizebox{!}{7.6in}{\includegraphics{#1}}% 
% }% 
% \newcommand\plottwo[2]{% 
%  \centering 
%  \leavevmode 
% % \includegraphics[width={\eps@scaling\columnwidth}]{#1}% 
%  \resizebox{.45\columnwidth}{!}{\includegraphics{#1}}%
%  \hfil 
% % \includegraphics[width={\eps@scaling\columnwidth}]{#2}% 
%  \resizebox{.45\columnwidth}{!}{\includegraphics{#2}}%
% }% 
% \newcommand\plotsmall[1]{% 
%  \centering 
%  \leavevmode 
%  \resizebox{.75\columnwidth}{!}{\includegraphics{#1}}% 
% }% 

%color/bandpass commands
%\newcommand\ubv{\mbox{$U\!BV$\ }}
%\newcommand\rbrv{\mbox{$R\!BR\!V$\ }}
%\newcommand\ub{\mbox{$U\!-\!B$\ }}
%\newcommand\vr{\mbox{$V\!-\!R$\ }}
%\newcommand\br{\mbox{$B\!-\!R$\ }}
%\newcommand\bv{\mbox{$B\!-\!V$\ }}

%journal title commands
%\newcommand\aj{\rmfamily{AJ}}%
%\newcommand\araa{\rmfamily{ARA\&A}}%
%\newcommand\apj{\rmfamily{ApJ}}%
%\newcommand\apjl{\rmfamily{ApJ}}% 
%\newcommand\apjs{\rmfamily{ApJS}}%
%\newcommand\aap{\rmfamily{A\&A}}%
%\newcommand\baas{\rmfamily{BAAS}}%
%\newcommand\mnras{\rmfamily{MNRAS}}%
%\newcommand\pasp{\rmfamily{PASP}}%
%\newcommand\procspie{\rmfamily{Proc.~SPIE}}%
%\newcommand\nat{\rmfamily{Nature}}% 

%\end{pirated/hacked commands}

\setlength{\oddsidemargin}{0.0in}
\setlength{\evensidemargin}{0.0in}
\setlength{\topmargin}{0.0in}
\setlength{\textwidth}{6.5in}
\setlength{\textheight}{9.0in}
\setlength{\headheight}{0.0in}
\setlength{\headsep}{0.0in}
\setlength{\marginparsep}{0.0in}
\setlength{\marginparwidth}{0.0in}
\setlength{\footskip}{0.0in}

\begin{document}
% Make pagestyle for succeeding frontmatter paginated, but without headers.
% See ``The LaTeX Companion,'' pages 91-95, for information on page styles.
\pagestyle{empty}

\begin{spacing}{1.1}
\begin{center}
Announcing the \\
Final Examination of \\
Scott Michael \\
for the \\
Degree of Doctor of Philosophy in Astronomy \\
Monday, August 4, 2003, 10:00 AM \\
Room 318, Swain Hall West
\end{center}
\end{spacing}

\begin{spacing}{1.0}
\vspace{0.3cm}
\noindent
Dissertation: {\it Planet Migration Induced by Gravitational Instabilities}
\vspace{0.3cm}

\noindent
The question of how gas giant planets form has been an area of active research for over a decade. Over the past several years, research of gas giant planet formation has yielded two major theories, core accretion and gravitational instability. Although some studies have indicated that gravitational instabilities cannot form gas giant planets directly, it is clear that they can have a significant effect on a protoplanetary disk. In this thesis I several possible effects that gravitational instabilities may have on a protoplanetary disk. This study is carried out using three dimensional numerical simulations of protoplanetary disks. Over twenty simulations with varying initial conditions, resolutions, and physical effects have been carried out and analyzed. Although all indications from these simulations are that gravitational instabilities cannot form gas giant planets directly at radii smaller than 40AU, I have learned that gravitational instabilities can have a dramatic effect on protoplanetary disk structure and planets embedded in a protoplanetary disk. 

Results include the {\bf key results here}

\vspace{0.3cm}

\begin{tabbing}
\textbf{Outline of Studies} \hspace{0.30\linewidth}
\= \underline{Educational Career} \kill
\underline{Outline of Studies} \> \underline{Educational Career} \\
Major:  Astronomy  \> M.A., 1999, Indiana University \\
Minor:  Physics    \> B.A., 1996, Wesleyan University
\end{tabbing}

%\hrule

\vspace{0.3cm}

\begin{center}
\underline{Committee in Charge} \\
\vspace{12pt}
Professor Richard H. Durisen, Chair (855-6921), Astronomy \\
Professor Stuart L. Mufson, Astronomy \\
Professor Haldan N. Cohn, Astronomy \\
Professor Charles J. Horowitz, Physics
\end{center}

\vspace{0.3cm}

\begin{tabbing}
Approved:\=\underline{\hspace{6.0cm}} \= \hspace{0.1\linewidth}
Approved:\underline{\hspace{6.0cm}} \kill
Approved:\>\underline{\hspace{6.0cm}} \> \hspace{0.1\linewidth}
 \\
\>Richard H. Durisen, Chair \> \hspace{0.2\linewidth}
\end{tabbing}

\vspace{0.3cm}

\noindent
(Any member of the Graduate Faculty may attend.  As a courtesy, please notify
the Committee Chair in advance.)
\end{spacing}
\end{document}
