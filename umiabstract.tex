\documentclass[letterpaper,11pt]{article}

% Aaron Steinhauer
% Indiana University Astronomy Department
% Email:  aaron@astro.indiana.edu

% Ph.D. Dissertation Announcement

%
%all the LaTeX formatting and definition crap
%for Aaron Steinhauer's thesis
%2003
%Stolen mostly from Adam Rengstorf and Shawn Slavin
%

\usepackage[figuresright]{rotating}
\usepackage{amsmath}
\usepackage{amssymb}
\usepackage{mhchem}
%\usepackage{doublespace}
\usepackage{setspace}
\usepackage{natbib}
%\usepackage{graphics}
%\usepackage{deluxetable}
%\usepackage{mydefs}
\usepackage{afterpage}
\usepackage{float}
%\usepackage{xspace}
\usepackage{verbatim}
\usepackage{graphicx}
\usepackage{grffile}
\usepackage{epstopdf}
\usepackage[font=small,format=plain,labelfont=bf,up,textfont=it,up]{caption}
\usepackage{subfigure}
\usepackage{geometry}   %SM
%\usepackage{xcolor}
%\usepackage{dcolumn}

\setcounter{secnumdepth}{3}
\setcounter{tocdepth}{2}


% Replaced setlengths with geometry call

\geometry{papersize={8.5in,11in},textwidth=5.95in,textheight=8.3in,left=1.525in,top=1.525in,headheight=0.15in,headsep=0.35in,marginparsep=0.125in,marginparwidth=0.75in,verbose}

% \special{papersize=8.5in,11in}
% \setlength{\oddsidemargin}{0.525in}
% \setlength{\evensidemargin}{0.525in}
% \setlength{\topmargin}{0.025in}
% \setlength{\textwidth}{5.95in}
% \setlength{\textheight}{8.3in}
% \setlength{\headheight}{0.15in}
% \setlength{\headsep}{0.35in}
% \setlength{\marginparsep}{0.125in}
% \setlength{\marginparwidth}{0.75in}
% \setlength{\footskip}{0.125in}

\makeatletter
\renewcommand{\@biblabel}[1]{\hspace{-\itemsep}}
%\let\ps@plain=\ps@headings
\setlength\@fpsep{2in}
\makeatother

\makeatletter
\renewcommand*\@makechapterhead[1]{%
  \vspace*{5\p@}%
  {\parindent \z@ \raggedright \normalfont
    \ifnum \c@secnumdepth >\m@ne
        \huge\bfseries \@chapapp\space \thechapter
        \par\nobreak
        \vskip 15\p@
    \fi
    \interlinepenalty\@M
    \Huge \bfseries #1\par\nobreak
    \vskip 35\p@
  }}
\renewcommand*\@makeschapterhead[1]{%
  \vspace*{5\p@}%
  {\parindent \z@ \raggedright
    \normalfont
    \interlinepenalty\@M
    \Huge \bfseries  #1\par\nobreak
    \vskip 35\p@
  }}
\makeatother

%\end{Thanks Shawn}

\def\citeapos#1{\citeauthor{#1}'s (\citeyear{#1})}
\renewcommand{\bibname}{References}
\renewcommand{\floatpagefraction}{0.5}
\renewcommand{\textfloatsep}{0.2in}
%\floatplacement{deluxetable}{tbp}
%\renewcommand{\HII}{\ion{H}{2}\xspace}    %  HII
\bibpunct{(}{)}{;}{a}{}{,}
%\newcommand{\feh}{$[Fe/H]$\xspace}
\newcommand{\dif}{\mathrm{d}}

%\newcolumntype{d}[0]{D{.}{.}{-1}}

%\newcommand\teff{$T_{eff}$\xspace}

%include figure commands
\def\eps@scaling{1.0}% 
\newcommand\epsscale[1]{\gdef\eps@scaling{#1}}% 
% \newcommand\plotone[1]{% 
%  \centering 
%  \leavevmode 
% % \includegraphics[width={\eps@scaling\columnwidth}]{#1}% 
%  \resizebox{\columnwidth}{!}{\includegraphics{#1}}% 
% }% 
% \newcommand\plotport[1]{% 
%  \centering 
%  \leavevmode 
% % \includegraphics[width={\eps@scaling\columnwidth}]{#1}% 
%  \resizebox{!}{7.6in}{\includegraphics{#1}}% 
% }% 
% \newcommand\plottwo[2]{% 
%  \centering 
%  \leavevmode 
% % \includegraphics[width={\eps@scaling\columnwidth}]{#1}% 
%  \resizebox{.45\columnwidth}{!}{\includegraphics{#1}}%
%  \hfil 
% % \includegraphics[width={\eps@scaling\columnwidth}]{#2}% 
%  \resizebox{.45\columnwidth}{!}{\includegraphics{#2}}%
% }% 
% \newcommand\plotsmall[1]{% 
%  \centering 
%  \leavevmode 
%  \resizebox{.75\columnwidth}{!}{\includegraphics{#1}}% 
% }% 

%color/bandpass commands
%\newcommand\ubv{\mbox{$U\!BV$\ }}
%\newcommand\rbrv{\mbox{$R\!BR\!V$\ }}
%\newcommand\ub{\mbox{$U\!-\!B$\ }}
%\newcommand\vr{\mbox{$V\!-\!R$\ }}
%\newcommand\br{\mbox{$B\!-\!R$\ }}
%\newcommand\bv{\mbox{$B\!-\!V$\ }}

%journal title commands
%\newcommand\aj{\rmfamily{AJ}}%
%\newcommand\araa{\rmfamily{ARA\&A}}%
%\newcommand\apj{\rmfamily{ApJ}}%
%\newcommand\apjl{\rmfamily{ApJ}}% 
%\newcommand\apjs{\rmfamily{ApJS}}%
%\newcommand\aap{\rmfamily{A\&A}}%
%\newcommand\baas{\rmfamily{BAAS}}%
%\newcommand\mnras{\rmfamily{MNRAS}}%
%\newcommand\pasp{\rmfamily{PASP}}%
%\newcommand\procspie{\rmfamily{Proc.~SPIE}}%
%\newcommand\nat{\rmfamily{Nature}}% 

%\end{pirated/hacked commands}

\setlength{\oddsidemargin}{0.0in}
\setlength{\evensidemargin}{0.0in}
\setlength{\topmargin}{0.0in}
\setlength{\textwidth}{6.5in}
\setlength{\textheight}{9.0in}
\setlength{\headheight}{0.0in}
\setlength{\headsep}{0.0in}
\setlength{\marginparsep}{0.0in}
\setlength{\marginparwidth}{0.0in}
\setlength{\footskip}{0.0in}

\begin{document}
\pagestyle{empty}
\begin{center}
\begin{spacing}{1.0}
Aaron Steinhauer \\
\vspace{0.09in}Department of Astronomy \\
\vspace{0.09in}Indiana University \\
%vspace{0.1in}

% Extra text for UMI Abstracts
\vspace{0.09in}
advisor: Constantine P. Deliyannis, Ph.D. \\
\end{spacing}
\vspace{0.2in}\underline {Formation and Evolution of the Open Cluster Lithium Gap}

%\vspace{0.25in}
\end{center}

%\vspace{0.5in}
%\noindent
Lithium plays an important role at the intersection of many vital fields in
astronomy, including cosmology and stellar structure and evolution.  In this
thesis I explore the evolution of the ``lithium gap,'' a dramatic over-depletion
of lithium seen in F stars of the Hyades cluster which strongly contradicts the strictures of
the Standard Theory.  The issue is examined from the heretofore unexplored angle
of the timing of its formation, with proposed solutions such as mass loss, diffusion
and slow mixing predicting different epochs for the depletion.  Four clusters
were chosen with ages intermediate to the Hyades and the Pleiades (whose F stars 
do not show such depletion): M34, M35, M39, and NGC 3532.  High resolution 
spectra were obtained of the lithium region using the Hydra and Hydra II 
multi-object spectrographs
on the WIYN 3.5m and CTIO 4m telescopes.  These same spectra, and independent
photometry, were used to determine fundamental parameters for these clusters as
a whole (age, metal abundance, and interstellar reddening) and for the individual
stars (effective temperature and surface gravity).  Results include a clear 
anti-correlation between rotational velocity and depletion in the gap whose slope
becomes steeper with age, 
and the determination of an early epoch of formation, with older clusters exhibiting
greater depletion in the gap.  It is also 
discovered that the gap covers a greater range in effective temperature than
had previously been known.  All of these results are consistent with the predictions
from stellar evolution theory that includes treatments of rotationally-induced mixing, 
and they also argue strongly against diffusion and 
mass loss.  Combined with a litany of evidence from the literature, there
can be little doubt that stellar rotation plays a major role in determining the
lithium abundance evolution of F stars.
%\vspace{0.5in}

%\begin{spacing}{1}
%\noindent
%\hbox{\hspace{2.5in} \underbar{\hbox{\hspace{3.25in}}}} \\
%\hbox{\hspace{2.5in} Constantine P. Deliyannis, Ph.D.}

%\vspace{0.5in}

%\noindent
%\hbox{\hspace{2.5in} \underbar{\hbox{\hspace{3.25in}}}} \\
%\hbox{\hspace{2.5in} R.~Kent Honeycutt, Ph.D.}

%\vspace{0.5in}

%\noindent
%\hbox{\hspace{2.5in} \underbar{\hbox{\hspace{3.25in}}}} \\
%\hbox{\hspace{2.5in} Catherine Pilachowski, Ph.D.}

%\vspace{0.5in}

%\noindent
%\hbox{\hspace{2.5in} \underbar{\hbox{\hspace{3.25in}}}} \\
%\hbox{\hspace{2.5in} Andrew Bacher, Ph.D.}
%\end{spacing}
\end{document}
