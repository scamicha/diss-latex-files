\chapter[Introduction]{Introduction}\label{chap:intro}
\thispagestyle{plain}
\begin{spacing}{0.9}
\begin{flushright}
{\it \footnotesize The White Rabbit put on his spectacles. \\``Where shall I begin, please your Majesty?'' he asked. \\
``Begin at the beginning,'' the King said gravely,\\ ``and go on till you come to the end: then stop.''\\}
 {\small -- Alice's Adventures in Wonderland, Chapter 12}
\end{flushright}
\end{spacing}

This work was born of the endeavor to answer the overarching question ``How do gas giant planets form?" The interest in this question was driven, in large part, by the discovery of the first extrasolar gas giant planet by \citet{mayor1995}. This discovery was soon followed by others and currently there are nearly 500 known extrasolar planets, or exoplanets. Few of these planets are terrestrial in nature, the vast majority are either gas or ice giant planets. The ubiquitousness of these objects reinforces the original question ``How do gas giant planets form?"

It has been theorized for over two centuries \citep{kant1755} that planets form from a disk surrounding a protostar. Modern treatments of this theory conclude that the formation of a star/disk system is a natural consequence of the collapse of a protostellar cloud \citep{cassen1981,yorke1993,vorobyov2006}. In fact, direct observations of circumstellar disks have existed for a decade \citep{padgett1999}. Clearly planets form from the gas and dust contained in the nebula, or protoplanetary disk; the point of debate now is the formation mechanism.

\section{Protostars and Protoplanetary Disk Systems}\label{sec:disks}

A protoplanetary disk is not an isolated system; it is instead a part of a forming stellar system. These systems are generally termed young stellar objects (YSOs), which include the protostar, the protoplanetary disk, and the infalling envelope. These objects are generally divided into three groups based on the infrared and millimeter excesses in their spectral energy distributions \citep{adams1987,kenyon1987,greene1994}; they are Class I, II, and III YSOs. Class I YSOs have large infrared excesses due to the fact that they are young embedded stars surrounded by a dusty disk and infalling envelope. Once the envelope has dissipated and only the disk remains, the YSO is characterized by moderate infrared excess and has become a Class II object. Class III YSOs have little infrared excess and are in a late phase of the formation process. Class III objects have either no disk or a very tenuous one, which produces little infrared emission. In addition to the Class I, II, and III objects, researchers have suggested a Class 0 phase, in which the star is so deeply embedded in the envelope that optical emission from the star is not visible \citep{andre1993}. Others have argued that proposed Class 0 objects may actually be Class I objects that have formed in a denser environment and therefore have more substantial envelopes \citep{jayaward2001}. 

Much of this work is applicable to T Tauri stars, a class of variable stars named after their prototype – T Tauri. They are pre-main sequence stars found near molecular clouds and identified by their optical variability and strong chromospheric  lines, and are typically surrounded by a cirumstellar accretion disk. T Tauri stars are classified into two categories: classical T Tauri stars (CTTS), which have excess infrared emission, strong line emission, and an active accretion disk, and weak line T Tauri stars (WTTS), which have little infrared excess, weak line emission, and a mostly dissipated disk. CTTS are considered to be Class I or II objects while WTTS are Class III objects. YSOs are thought to follow an evolutionary sequence through the different Classes, and therefore transition from CTTS to WTTS. Although there are some indications that the Class I to II transition can occur in $\sim1$ Myr \citep{eisner2006}, firm estimates on the duration of each of the phases are not well known, and probably vary from one system to another. 

\section{Gas Giant Formation Theories}\label{sec:theory}

There are two major theories of gas giant planet formation that are
currently being considered -- core accretion, or nucleated instability, and disk instability. The nucleated instability theory, first discussed by \citet{safronov1969} and later improved by \citet{mizuno1980}, has come to be considered the standard theory of gas giant planet formation. It states that a rocky core builds up via accretion of solids in the protoplanetary disk. When this core is massive enough, it will accrete the surrounding gas until it clears a gap within the disk. The disk instability theory first proffered by \citet{kup1951} and later revisited by \citet{boss1997} postulates that protoplanetary disks susceptible to gravitational instabilities (GIs) could form gravitationally bound clumps, which would eventually contract to become gas giant planets. 

Each of the theories has advantages and challenges; in the next sections, I will highlight some of them. It should be noted that hybrids of the two theories has been proposed (see \S\ref{sec:Intro:disktheory} where the nucleated instability is accelerated by the action of GIs.

\subsection{The Nucleated Instability Theory}\label{sec:coretheory}

In the nucleated instability picture, a large solid core must first build up through the collision of rocky materials \citep{safronov1969}. When this core grows to several Earth masses, it gravitationally attracts the surrounding gas to form a gas giant planet \citep{mizuno1980}. Observational estimates of disk lifetimes range from 0.1 to 10 Myr, \citep{haisch2001,chen2004}. Simulations by \citet{pollack1996} found that core accretion could produce a gas giant planet $\sim 8$ Myr after kilometer-sized planetesimals have formed. In addition, \citet{pollack1996} found that the formation time is highly dependent on the initial protoplanetary disk surface density. Figure \ref{fig:pollack} reproduces portions of Figure 1 and Figure 2 from \citep{pollack1996} and shows the dramatic variation of formation time with initial surface density. However, the core masses ($M_{core}$) of the planets produced by these simulations ($\sim 15 M_{\oplus}$) are outside the range of Jupiter's core mass (0--11 $M_{\oplus}$) inferred from the Jovian gravitational moments and detailed equation of state calculations \citep{saumon2004}. The large range of possible $M_{core}$ is due to uncertainties in the equation of state (EOS) of hydrogen and helium at megabar pressures. \citet{saumon2004} found the most likely values of $M_{core}$ to be $\leq 5 M_{\oplus}$. However, new simulations by \citet{militzer2008} predict a Jupiter core mass of 14--18 $M_{\oplus}$. The results of \citet{saumon2004} allow for several possibilities: (1) Core accretion can occur for $M_{core} \leq 5 M_{\oplus}$, (2) Jupiter formed with a more massive core which was then dispersed into Jupiter's outer layers, or (3) Jupiter did not form via core accretion. More recent work by \citet{hub2005}  and \citet{papa2005} have addressed the first point by demonstrating that a gas giant planet can form via core accretion with $M_{core}=5 M_{\oplus}$ in 4.5 Myr and 3 Myr, respectively. However, for these simulations, the authors used dust grain opacities of 2\% and 1\% of the interstellar value. When the dust grain opacity is set to the interstellar value, gas giants form in 95 Myr and 30 Myr. The critical uncertainty in determing these timescales is the growth time of the protoplanet's envelope. \citet{ida2008} recast the equation for the growth timescale of a protoplanet's envelope $\tau_{e} = M(\mathrm{d}M_{e}/\mathrm{d}t)^{-1}$ to 
\begin{equation}
\tau_{e} = k_1 \left(\frac{M_p}{M_\oplus}\right)^{-k_2} \mathrm{yr},
\end{equation}
where $M_p$ is the planetary mass, including the envelope, and $M_\oplus$ is an Earth mass. $k_1$ and $k_2$ are parameters which have strong dependence on several factors, including the envelope opacity. Values for the \citet{pollack1996} simulations are $k_1=10^{10}$ and $k_2=3$.

\begin{figure}[p]
\begin{center}
%\unitlength1in
\includegraphics[scale=0.60]{figures/chap1/pollack_plots.eps}
\caption[Figures from \citet{pollack1996}.]{Panels from Figures 1 and 2 of \citet{pollack1996}. Each panel represents the growth of a gas giant planet via the nucleated instability. The solid line represents the growth of solids, the dotted line the accretion of hydrogen and helium, and the dashed line represents the total mass. Three simulations are shown with the principle difference being the surface density of solids at Jupiter's radius, which varies between 7.5 $\mathrm{g/cm}^2$ and 15 $\mathrm{g/cm}^2$. Clearly the surface density of solids has a strong effect on growth time. Figure reproduced with permission of co-author Olenka Hubickyj.}
\label{fig:pollack}
\end{center}
\end{figure}

Forming kilometer-sized planetesimals needed to build the cores may be difficult due the effects of migration. Planetesimals must grow from small dust grains in the gas disk which is accreting into the central star. Since the particles are strongly affected by gas drag until they reach sizes of tens of meters \citep{weiden1977}, they must grow to this size quickly or be accreted into the star. See \S\ref{sec:migtheory} for a further discussion on the effects of migration.

\subsection{The Disk Instability Theory}\label{sec:Intro:disktheory}

On the other hand, disk instability \citep{kup1951,cameron1978,boss1997,durisen2007} can form dense clumps via GIs in times comparable to the dynamic time of the disk ($\lesssim 10^3$ yr). According to the theory, these clumps are the self-gravitating precursors to gas giant planets. As the clump contracts, the solids contained in the gas will rain out to form a core \citep{slattery1980,helled2008a,helled2008b}, and the clump can capture more solids from the surrounding disk \citep{helled2006,guillot2010}. A protoplanetary disk must meet three main requirements to form gas giant planets via disk instability: (1) GIs must occur in the disk, (2) the GIs must cause the disk to fragment into dense clumps, and (3) these clumps must survive long enough to become gravitationally bound. 

To parameterize the susceptibility of a disk to instability, we consider the Toomre $Q$ parameter \citep{toomre1964}, where 
\begin{equation}
Q = \frac{c_s\kappa}{\pi G\Sigma}.
\label{eq:toomreq}
\end{equation}
Here, $c_s$ is the sound speed, $\kappa$ is the epicyclic frequency, where $\kappa$ is calculated by $\kappa = \sqrt{r d\Omega^2/dr + 4\Omega^2}$ using azimuthally averaged midplane values of $\Omega$, and $\Sigma$ the gas surface density. 
These factors in equation \eqref{eq:toomreq} represent stabilizing and destabilizing influences in the disk. The local pressure, represented by $c_s$, acts to stabilize short wavelengths while the disk rotation, represented by $\kappa$, acts to stabilize long wavelengths. Disk self-gravity, represented by $\pi G \Sigma$, acts to destabilize the disk. 
For $Q \leq 1$, a disk is highly unstable to an axisymmetric ring instability \citep{toomre1964}, and spiral instabilities set in at somewhat higher $Q$-values. Numerical simulations generally show that disks with $Q$-values $\sim 1.5$ to 1.7 are marginally unstable \citep{boss2000,pickett2003,durisen2007}.  

Even though spiral disturbances due to GIs may exist in a disk, they may not fragment into dense clumps. Whether a disk will fragment is largely controlled by the cooling time. Isothermal calculations of low-Q disks, representing instantaneous cooling, tend to form many fragments \citep{boss2000,pickett2003}. Simulations using parameterized cooling find that the cooling times, with an adiabatic index, i.e. the ratio of specific heats, $\gamma = 5/3$ for the gas, must be less than about half to one orbit period for fragments to appear \citep{gammie2001,rice2003b,mejia2005}. For a $\gamma = 7/5$ gas,  fragmentation occurs for cooling times as long as two to four disk orbit periods \citep{rice2005,clarke2007}. Regardless of the EOS, simulations using more realistic radiative physics find that cooling times are too long to form fragments  in protoplanetary disks inside about 40 AU \citep{cai2006,cai2008,boley2006,boley2007b,boley2008,stamatellos2008,forgan2009}. Although there are some dissenters \citep{boss2007,mayer2007}, the failure of realistically cooled disks to fragment inside 40 AU is strongly supported by analytic arguments \citep{rafikov2005,rafikov2007}.
On the other hand, recent analytic or semi-analytic work \citep{clarke2009,rafikov2009,dodson2009} and simulations \citep{stamatellos2009,boley2009,boley2010} indicate that disk fragmentation and gas giant protoplanet formation may indeed occur in real disks outside 100 AU. 

Once fragments form, they must survive long enough to become bound by self-gravity. Although many clump forming simulations have found clump lifetimes of many disk orbits \citep{boss2003,rice2003b,mayer2004}, other simulations find that the clumps are destroyed by the shearing motion of the disk \citep{pickett2003,mejia2005}. Although some of this discrepancy may be due to different numerical methods \citep{pickett2007}, simulations using the same code show clump destruction and survival also depend on the physical conditions in the disk. Obviously the particulars of a disk (EOS, mass, temperature, opacity, location, etc.) play a pivotal role in all aspects of the disk instability theory. Given the existing difficulties with both theories, some authors have proposed hybrid theories where GIs assist planetesimal formation and accelerate core accretion \citep{hag2003b,rice2004,rice2006,durisen2005}.

\section{Other Effects of Gravitational Instabilities}\label{sec:xtraeffect}

Regardless of whether planets typically form via disk instability, nucleated instability, or some combination, gravitational instabilities can play a key role in various aspects of protoplanetary disk evolution. Although there are convincing analytical arguments \citep{rafikov2005,rafikov2007} and simulation results showing \citep{nelson2000a,cai2006,boley2006,boley2007b,stamatellos2008,boley2008} that protoplanetary disks can not undergo fragmentation via gravitational instability at radii $\lesssim 40$AU, analytical arguments and theoretical modeling has shown that GIs can be active for a range of disk parameters over a large range of radii \citep{vorobyov2008,vorobyov2009,rafikov2009,zhu2009}. Indeed, it is generally accepted that Class I objects are massive enough CTTS to be GI active, additionally GIs could influence FU Orionis outbursts in this stage of evolution \citep{armitage2001,vorobyov2006,boley2008,zhu2009}. However, this may not be the only phase in disk evolution when GIs can play a major role \citep[c.f.][]{zhu2010} since GIs are not only strongly affected by mass, but also by disk energetics. This makes auxiliary effects of GIs particularly interesting since they may be ubiquitous. In non-fragmenting disks, GIs can play an important role in mass and energy transport, the migration of embedded massive objects, and the motion of the central star. I will explore each of these throughout this work.    

% This work does not focus so much on the debate about planet formation mechanisms, but on possible effects of GIs beyond protoplanetary disk fragmentation. This is mainly due to the fact that a consensus is beginning to emerge that although most{\bf need a citation}, if not all, protoplanetary disks go through a phase where they should be susceptible to GIs, it is unlikely that the outcome of these instabilities is direct formation of gas giant planets at radii $\lessim 40$AU \citep{nelson2000a,cai2006,boley2006,rafikov2007,boley2007b,stamatellos2008,boley2008}. However, even if GIs do not form gas giant planets, they may still affect a protoplanetary disk and any embedded gas giant planets or gas giant planet embryos in significant ways. The key effects that I will be exploring throughout this work include mass transport, migration of embedded massive objects, stellar wobble, and shepherding of small objects.

\begin{spacing}{1.0}
\subsection{Understanding the Impact of Numerical and Physical Effects on GIs}\label{sec:under}
\end{spacing}

To date, much effort has been devoted to understanding when and how GIs can cause a disk to fragment. Less emphasis has been placed on understanding how various disk properties might change the initiation of GI activity and what effect such properties might have on the basic characteristics of the instabilities. Key questions such as ``What are the primary factors that determine the dominant unstable mode in a GI active disk?" or ``How do physical quantities in the disk (e.g. surface density profile, composition, equation of state) affect the onset and growth of GIs?" still have no definitive answer. These questions and their corollaries are important when considering effects of GIs beyond direct planet formation. In chapter \ref{chap:compare} I address some of these issues, by varying both physical and numerical parameters for several simulations. Analysis of these disks provides insight into physical and numerical factors that can dramatically alter the  character of GIs. Specifically, I study the effectiveness of GIs as a mass transport mechanism and attempt to determine whether angular momentum transport in a GI active disk is a local or global process \citep{balbus1999,vorobyov2010}.

\subsection{Stellar Motion Caused by GIs}\label{sec:wobbletheory}

\citet{adams1989} proposed that a $m = 1$ pattern in a circumstellar disk could grow via interaction with the central star. This amplification, termed Stimulation by the Long-range Interaction of Newtonian Gravity, or SLING, acts when an $m = 1$ mode is stimulated and reflects from the outer disk edge. When reflected, trailing waves become leading waves and vice versa. The amplification occurs if the reflected waves have the same phase as those generated by resonant forcing of the central star \citep{shu1990}. Even if SLING amplification is not acting, $m = 1$ patterns can develop from the interaction of higher-order modes. This second-order effect can still be substantial if the interacting modes are highly non-linear. Some numerical work has been done to study the possibility of stellar motion caused by waves created from GIs, but it has either focused on fragmenting disks \citep{boss1998} or been dominated by local modes \citep{rice2003a}. In chapter \ref{chap:indirect}, I examine the effect of GI activity on stellar motion.

\subsection{Planet Migration}\label{sec:migtheory}

Regardless of how a gas giant planet forms, whether by core accretion, gravitational instability, or a hybrid of the two, an interesting question is what happens to a protoplanet after it has formed and before the disk dissipates.  Linear analysis \citep{ward1997,tanaka2002} as well as numerical simulations \citep{nelson2003,nelson2004} indicate that in a laminar disk, or a disk dominated by MHD turbulence, 
planets tend to migrate inward via Type I or Type II migration and may be accreted onto the central star. The distinction between Type I and Type II migration is whether or not the planet can form a gap in the disk. If no gap forms, the planet undergoes Type I migration where the planet and disk exchange angular momentum via gravitational torques. In the case of Type II migration, a gap forms and the planet migrates inward due to the overall disk evolution caused by viscous stresses. Type III migration occurs for objects in the transition between Type I and Type II migration, i.e., bodies massive enough to create a dip or gap in the surface density profile locally but not massive enough to completely clear a gap. In the case of Type III migration, fluid elements follow horseshoe streamlines in the vicinity of the planet. When the fluid elements execute a U-turn at the end of the streamlines, they can gain or lose angular momentum from the planet, resulting in a coorbital corotation torque \citep{masset2003}. Because this is a transitional case, and the streamlines that give rise to the coorbital torque are too complex and on too small of a scale to model accurately with a global fixed grid, I do not consider Type III migration in my investigations. This description is included for completeness. 

\subsubsection{Gap Formation}

The gravitational torque exerted by an embedded object on the disk is
\begin{equation}
T_g=\zeta r_p^4\Omega^2\left(\frac{r_p}{\Delta r}\right)^3\left(\frac{m_p}{M_\star}\right)^2.
\end{equation}
Here $r_p$ is the orbital radius, $\Omega$ the orbital frequency, $m_p$ the planet mass, $M_\star$ the star mass, $\Delta r$ is the larger of the disk scale height, $H$, and the Hill radius of the planet, $R_H=r\left(\frac{m_p}{3M_\star}\right)^{1/3}$, and $\zeta$ is a numerically determined factor $\sim 0.25$ \citep{lin1979b,lin1979a,goldreich1980,lin1986}. This torque acts to transfer angular momentum from the planet to the disk at $r_p + \Delta r$ and to transfer angular momentum from the disk to the planet at $r_p - \Delta r$. This causes material at larger radii to gain angular momentum and move outwards while material inside the planet's orbit will lose angular momentum and move inwards. The net effect is for a gap to open in the disk at the planet's orbital radius. This may be counterbalanced by other mechanisms which act as viscous stresses. The viscous torque exerted on material outside $r$ by material inside $r$ in a keplerian disk is \citep{lyndenbell1974},
\begin{equation}
T_v = 3\pi \Sigma \nu \Omega r^2.
\label{eq:tvisc}
\end{equation}
This torque acts to transfer angular momentum outward and mass inward. If it is greater than the gravitational torque at $r+\Delta r$, then material will move inward and gap formation will be halted. So the condition for gap formation is $T_g > T_v$.

If one adopts the $\alpha$-disk prescription described by \citet{shakura1973} (see \S \ref{sec:IC:alphadisk}) and assumes $\nu = \alpha c_s H$ where $\alpha$ is a parameter and $c_s$ is the isothermal sound speed, then the torque inequality implies that 
\begin{equation}
\left(\frac{m_p}{M_\star}\right)^2 \gtrsim 3 \pi \zeta \alpha \left(\frac{H}{r}\right)^5
\label{eq:gapform1}
\end{equation}
must be satisfied to form a gap if $H>R_H$. If $H<R_H$ then the criterion becomes,
\begin{equation}
\frac{R_H}{r} \gtrsim \zeta \alpha.
\label{eq:gapform2}
\end{equation}
See Figure \ref{fig:gap} for a comparison between a gap forming planet and a non-gap forming planet.

\begin{figure}[p]
\centering
\unitlength1in
\begin{minipage}[t]{6.3in}
\centering
\includegraphics[scale=1.275]{figures/chap1/nogap.eps}
\end{minipage}\\
\begin{minipage}[t]{6.3in}
\centering
\includegraphics[scale=1.275]{figures/chap1/gap.eps}
\end{minipage}
\caption[Examples of an object that does not form a gap and one that does]{These images taken from \citet{chambers2009} show examples of numerical simulations that contain an object that does form a gap (bottom) and one that does not (top). Figure reproduced with permission of author Pawel Artymowicz.}
\label{fig:gap}
\end{figure}

\subsubsection{Type I Migration}
In the case of Type I migration, the protoplanet is not massive enough to open a gap in the disk and migrates inward by exchanging angular momentum with the disk via gravitational torques. Typically in Type I migration, the mass of the planet is small enough that its gravitational perturbations do not have a large effect on the structure of the surrounding disk. However, for a given spiral mode the planet can couple strongly to the disk at the inner and outer Linblad resonances \citep{papaloizou2007}.  The planet potential can be Fourier decomposed in the azimuthal or $\phi$-direction. Each component can be thought of as a perturbing potential $\Phi_m(r,\phi)$, which has azimuthal mode number $m$ and rotates with pattern frequency $\Omega_p$ at an angular speed of $\Omega$. In this case, angular momentum is exchanged with the disk whenever $m(\Omega -\Omega_p)$ equals either 0 or $\pm\kappa$. For a Keplerian disk $\kappa \equiv \Omega$ so this occurs at the corotation resonance, when $\Omega = \Omega_p$. It also occurs at the inner Linblad resonance, when $\Omega = \Omega_p +\Omega/m$, and the outer Linblad resonance, when $\Omega = \Omega_p -\Omega/m$. Spiral density waves are launched at the Linblad resonances and transfer angular momentum throughout the disk. The waves at the outer Linblad resonance produce a negative torque (i.e., transfer angular momentum from the planet to the disk), while waves at the inner Linblad resonance produce a positive torque (i.e., transfer angular momentum from the disk to the planet). In general, these torques are not equal, partly due to the fact that the outer Linblad resonance lies closer to the planet than the inner one.The result is a net decrease of angular momentum for the planet \citep{papaloizou2007}. This, in turn, leads to the inward migration of the planet.

For Type I migration, the timescale for the protoplanet to migrate into the central star can be less than a few $\times 10^5$ years, short compared to disk lifetimes. In an isothermal disk, the migration rate is approximately,
\begin{equation}
\frac{\mathrm{d}r_p}{\mathrm{d}t} \approx -(2.7 + 1.1p)\left(\frac{M_p}{M_\star}\right)\left(\frac{\Sigma r_p^2}{M_\star}\right)\left(\frac{c_s}{r_p \Omega_p}\right)^{-2}\Omega_p.
\label{eq:migrate}
\end{equation}
Here, the gas surface density $\Sigma \propto r^{-p}$, $\Omega_p$ is the angular orbital velocity of the disk at the planet radius, and $c_s$ is the isothermal sound speed \citep{tanaka2002}. This result, derived using linear analysis, has been confirmed via numerical simulations for planet masses less than a few times $M_\oplus$ \citep{dangelo2003,dangelo2008}.

Although this rapid migration timescale is problematic for the survival of growing gas giant planet cores, there are many factors that can alter Type I migration. Investigations into how Type I migration might proceed differently and change the migration rate have had some success. For example, most work done on Type I migration assumes a vertically isothermal disk structure, but recent work has shown that Type I migration may be altered substantially for a non-isothermal disk \citep{paardekooper2006,kley2008}. In fact, under certain disks conditions, planet cores may migrate {\it outward} \citep{kley2008}. Additionally, if the disk is gravitationally unstable, the instabilities may slow inward migration, or even cause the protoplanet to move outward. These effects may help to explain how embryos, cores, and protoplanets can survive in a protoplanetary disk after they form as well as provide insight into their distributions in multiple planet systems, more than fifty of which have already been discovered. So far, migration in gravitationally unstable disks has only been studied in the context of fragmented disks \citep{mayer2004,boss2005}.


\subsubsection{Type II Migration}

With Type II migration, the protoplanet is massive enough to open a gap and then migrates with the motion of the surrounding disk. Once the gap has been opened the planet's migration rate is tied to the viscous evolution of the disk. The planet migrates inward at the same rate as the gas inflow rate due to viscosity. This rate is 
\begin{equation}
\frac{\mathrm{d}r_p}{\mathrm{d}t} =-\frac{3 \nu}{2 r_p},
\label{eq:type2mig}
\end{equation}
where $\nu$ represents the viscous stress in the disk.

If the planet moves toward the inner or outer edge of the gap, the resulting torque imbalance acts to restore its position to the middle of the gap. Although the timescale for the viscous evolution of the disk is generally shorter than the Type I migration timescale, it is still sufficiently short to pose problems for planet survivability. Very massive planets undergo migration at a reduced rate compared to equation \eqref{eq:type2mig} due to the fact that there is a maximum torque that can be exerted by the disk from its viscous evolution. When the planet mass $M\gtrsim \pi r_p^2\Sigma$, the migration rate depends on $\Sigma r_p^2/M$ instead of depending solely on the viscosity. 

\section{This Work}

In this thesis, I investigate several of the aforementioned phenomena caused by GIs in protoplanetary disks through three dimensional radiative hydrodynamic simulations. By varying several different parameters and simulating new physical mechanisms, I explore the effects of varied conditions on GIs. I also study how GIs can impact stellar motion and planetary migration. In all, I am seeking to answer the following questions:
\begin{itemize}
\item How do the following affect GI onset and evolution:
\begin{itemize}
\item Surface density
\item Azimuthal resolution
\item Equation of state
\item Initial random perturbation
\end{itemize}
\item What is the interplay between GI activity and stellar motion?
\item How do GIs affect planet migration?
\item Can planets or protoplanets trigger GIs?
\item How does the presence of a planet or protoplanet affect the onset or characteristics of GIs? 
\end{itemize}

In chapter, \ref{chap:numeth} I discuss the various aspects and features of the hydrodynamic codes used. Specifically, I describe the inclusion of the indirect potential method and the Verlet integration method to include stellar motion and planetary companions. Chapter \ref{chap:compare} looks at the ramifications of varying several physical and numerical parameters in disk simulations, including initial surface density profile, equation of state, initial density perturbation, and numerical resolution. Detailed analysis of disk dynamics, mass transport rates, and GI characteristics are performed. In chapter \ref{chap:indirect}, the outcome of allowing the central star to move in a GI active disk is examined and compared to the fixed star case. Chapter \ref{chap:planet} studies the effect of GIs on planetary migration and also examines how the presence of a planet can impact the onset of GIs. Finally, in chapter, \ref{chap:conclusion} I summarize the key results. 
